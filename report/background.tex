\chapter{Background}

\section{Solar water heating systems}

\subsection{Need and uses}

As we enter the twenty-first century, sustainability, energy security, and new ways to meet our increasing energy demands are on the agendas of many governments.
In Australia, water heating is the second-largest residential energy end use, accounting for an estimated 23\% of energy consumption, according to the Commonwealth government's Residential End-Use Monitoring Program \cite{REMP12}.
This is second only to HVAC (air conditioning and space heating).
In some nations, the number may be as high as 30\% to 50\% \cite{Lane96}.
Therefore, using `free' solar energy to heat water is an obvious way to significantly reduce domestic power consumption.

Solar domestic hot water (SDHW), or solar hot water systems (SHWS) are \opinion{something}.

\subsection{Typical system}

Describe a typical solar hot water system here.

\subsection{Economics}

Fitzmorris performed an economic analysis of the current SDHW industry in North America \cite{Fitzmorris10}.
He found that although solar water heating is positioned to be a disruptive technology, due to its superior energy efficiency compared to traditional water heating, it was actually losing market share due to problems in supply chain and the increased up-front cost of these systems.
He identified two important markets: construction firms who want to minimise the cost of a new house, and homeowners who need to replace a broken heating system.
The former is sensitive to the highly increased price of the solar system (Fitzmorris estimates up to \$2000 for a solar system installation, versus \$300 for an electric heater).
The latter market is sensitive to the increased complexity of installing a solar system - there is more design work involved in properly sizing and installing a solar water heater currently, and replacement speed is typically a deciding factor.

Once a solar water heating system is running, though, the gains are found to be significant.
Anecdotally, the University of Newcastle reported a nearly-80\% reduction in energy use (and energy bill) in a six-storey teaching and research building \cite{ApricusNewcastle}.
Less anecdotally, members of the School of Energy and Power Engineering at the Xi'an Jiaotong University designed and studied a SHWS for a hotel in the city \cite{Cao14}.
They considered a replacement for the hotel's current gas geyser water heater, and predicted that it would have a payback period of just 7.4 years, as well as a lower {\it total} investment over its 20 year lifetime than the {\it initial} amount spent on the gas geyser heater.
This analysis included the benefits of carbon credits the hotel would receive from the government.

\subsection{Control}

Current hot water systems typically use simple temperature-differential controllers to force water through the solar collectors and to decide when to use a secondary booster to heat the tank water.
This approach is outlined in Sustainability Victoria's handbook for solar thermal system design \cite{LSTS}.
The handbook points out that simple controllers such as timers or single sensors cannot adequately react to the entire system state - for example, variations in insolation across a day, or variations in the storage temperature of the tank.
For this reason, it recommends the use of temperature differential controllers.

Simply, the temperature of the water exiting the solar collectors is compared to the temperature in the tank, and the pump is activated if there is a positive difference (i.e. useful heat is available).
The handbook goes on to detail refinements to this approach - using proportional motor speeds instead of binary control, hysterisis to avoid pump hunting (rapid on/off motor cycles), and also mentions integrating with building automation systems and ``time-of-day clocks ... for applications that have a repeated daily hot water demand pattern.''
