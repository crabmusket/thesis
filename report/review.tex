\chapter{Related and prior work}

\section{Model-predictive control}

\section{Convex optimisation}

\subsection{Mathematical background}

Convex optimisation refers to the class of mathematical optimisation problems
where the objective and constraint functions are all convex with respect to the
decision variable \cite{Boyd04}. The particular shape of convex optimisation
problems admits the use of very efficient algorithms to compute globally
optimum solutions.

\subsection{Applications}

When applied to an MPC controller in \cite{Wang10}, a bespoke quadratic program
(QP) solver achieved a 200Hz control frequency on a system with 13 states and 3
inputs over a 30-step control horizon. This was achievable by taking advantage
of the structure of the MPC problem and the QP that was formed. The authors
implemented an algorithm that solved the problem with time complexity linear in
the horizon dimension, rather than cubic as seen in interior-point methods
\cite{}.

\subsection{Convex relaxation}

\subsection{Disciplined convex programming}

While the theoretical advantages of using convex optimisation are widely known
\cite{Luo06}, in practise these rewards may be difficult to reap. While there
exist general-purpose solvers for non-smooth convex optimisation problems which
have attractive theoretical properties, they may in practise be slower than
transforming a non-smooth problem into a smooth problem of higher dimension,
and using an efficient solver for the new problem class. Grant and Boyd opine
in \cite{Grant08} that this process is difficult and error-prone, even for
experts in the field. They propose a method called {\emph disciplined convex
programming} (DCP) \cite{Grant06} which aims to express convex optimisation
problems in such a way that transformation to an efficiently-solvable
representation can be automated, removing this ``expertise barrier'' as they
put it.

Recent DCP software such as CVX for MATLAB \cite{CVX} and CVXPY \cite{CVXPY}
implements this paradigm to provide a simple user interface without sacrificing
the efficiency of specialised solvers for subclasses of convex problems.
\opinion{It has yet to be seen whether the same gains discovered in
\cite{Wang10} in the area of MPC can be achieved by this general software.}
