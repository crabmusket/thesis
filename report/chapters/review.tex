\chapter{Related and prior work}
\label{ch:review}

\section{Model-predictive control}

\subsection{Robust MPC}

Robustness is the property that a controller functions well under various disturbances or environmental conditions.
In the context of a hot water system, a robust controller might ensure that there is always some hot water in the tank to satisfy unexpected loads, for example.
We are interested in robustness as it pertains to keeping the user satisfied in most conditions, which is the primary goal of a hot water service, and the reason thermostat control is attractive: it is pessimistic and provides the best possible robustness, at the cost of possible unnecessary heating.

\textcite{Jalali06} take the view that robust MPC has the goal of ``[minimizing] some objective function, while guaranteeing a set of constraints to be satisfied for all possible uncertainties.''
\textcite{Oldewurtel10} instead suggest that robustness can be formulated in terms of probability.
They note that in building automation problems, assigning hard constraints can often result in infeasible problems, so they instead create an objective function that represents the probability of constraints being violated.
Minimising this provides a robust response to disturbances modelled as Gaussians, which have ``infinite support'': a non-zero probability of occurrence at all values.
This approach has the additional advantage that European standards are specified in this way (probability of violating bounds), so the control design maps almost directly onto standards compliance.
This is a key benefit for practical implementations of MPC, where the controller must be designs in accordance with local and national regulations.

\textcite{Parisio14}, however, argue that the assumption of Gaussian distribution of disturbances is restrictive.
They retain the chance constraints and objective formulation from \authors{Oldewurtel10} but considered the maximum possible disturbance from a finite set of disturbance scenarios as a constraint.
These disturbance scenarios are hand-constructed as black-box functions that ``do not require the specification of particular probability distributions \dots but only the capability of randomly extracting from them''.
This was augmented using a vector of slack variables, which in effect assign a large penalty to constraint violation instead of declaring the problem infeasible.

\subsection{High performance/embedded hardware}

We are interested in applications of MPC that are implemented on constrained hardware.
A typical domestic hot water system may be designed with very minimal computer hardware, such as an embedded microcomputer.
Demanding requirements like this have historically prevented MPC for being practical in embedded applications but this may be changing.

When applied to an MPC controller by \textcite{Wang10}, a bespoke quadratic program (QP) solver achieved a 200Hz control frequency on a system with 13 states and 3 inputs over a 30-step control horizon.
This was achievable by taking advantage of the structure of the MPC problem and the QP that was formed.
The authors implemented an algorithm that solved the problem with time complexity linear in the horizon dimension, rather than cubic as seen in interior-point methods.
Note that these results were observed running on a 3GHz Athlon desktop processor, but a hot water system would require significanly slower control than 200Hz.

This is the sort of expert optimisation that Grant and Boyd would hide in the backend of DCP software, quietly applying this transformation if the problem fits it.
However, it remains to be seen whether intelligent transformations like this actually can be entirely automated and give such performance gains.

Another approach to high-performance MPC is code generation, in which a high-level program generates low-level code to solve a particular problem class, which can be run efficiently on the target platform.
According to \textcite{Mattingley12}, code generation can sidestep the need for complex, proprietary software, and targets cases where high performance is required.
In one example, \textcite{Vukov12} showed that the code generation of the ACADO nonlinear control tool~\cite{ACADO}, which outputs optimised C code, can achieve solve times averaging less than 1ms on a quad-core desktop processor.
\todo{More detail here.}

\section{Hot water system modelling}

\authors{Camacho04} estimate that one of the key differences in MPC control is the significant up-front modelling effort, as the system model is very important to the overall control achieved.
Therefore care must me taken to study various efforts in modelling hot water tanks.
In \autoref{ch:models} a model is developed following the work presented in this section.
The model must be able to adequately capture the subtleties of the domestic hot water tank in order to better evaluate the effect of different control strategies.
Some of the details outlined here will also be used to derive a simplified tank model used in the controllers themselves, as control of the full nonlinear model is beyond the scope of this thesis.

\textcite{Vrettos12} note that many current publications use ``single-point'' or zero-dimensional models of domestic water heaters, where the entire tank is represented by a single value (temperature or energy).
We refer to these models as mixed-tank models, as they treat the entire tank as a single uniformly-mixed body of water.
The alternative is a one-dimensional model, in which the tank is considered to be at uniform conditions across every horizontal section, but may vary at different depths.
This type of model is able to capture the effect of stratification, described in \autoref{sec:background:stratification}, which is an important factor in predicting the performance of hot water devices.

\subsection{Mixed-tank models}
\label{sec:review:mixed-tank}

\textcite{Halvgaard12} made use of a mixed-tank model in order to perform economic model-predictive control of the heating elements.
Their model involves a single heat balance equation,
$$ \dot{T} \propto Q_h + Q_s - Q_l - UA (T - T_{\text{amb}}). $$
The three heat flows, $Q_h$, $Q_s$ and $Q_l$ represent energy input/loss from the submerged heating element, solar collector, and load, respectively.
The last term accounts for losses to the environment.
The heat gained from the element is a constrained input, the solar heat gain is a function of the instantaneous insolation, and the load is based on predicted user activity.

Their model was verified by measuring the temperature in a large water tank at eight different vertical locations and averaging these readings to calulate the approximate energy in the tank.
In a practical experiment they found that even with this simplified model of the tank, their controller was able to reduce the operating power bill by up to 30\%.
This is further discussed in \autoref{sec:review:mpc:halvgaard}.

This experiment is notable for its large tank (790L whereas a typical tank in Australia might be half that size).
It must be pointed out that the authors explicitly mention that the tank has internal stratification-enhancing devices like those described in \autoref{sec:background:stratification}, even though the tank model is fully mixed.
It appears 

\Autoref{sec:review:mpc:halvgaard} details the portions of this paper relevant to control.

\subsection{Stratified models}
\label{sec:review:stratified-tank-models}

\textcite{Hollands89} review mathematical models for stratified tanks and note that it was usual to model the tank as several fixed-volume, variable-temperature layers.
\textcite{Kleinbach93} refer to this approach as the \emph{multinode} model, which they contrast to the \emph{plug flow} model, in which there are a variable number of layers, each with variable volume and temperature.
With as few as three of these \emph{multinode} layers, \authors{Hollands89} found that models could under-predict system performance by 10\%.
With complex supply and load patterns, they found that up to sixty-four layers were needed to capture the system's dynamics.

\textcite{Cristofari02} formulated a stratified tank model for simulation, rather than optimisation, of SWHSs in Corsica.
They assume that a perfect diffuser exists on the tank inputs and outputs, so that water added or removed is done so from an appropriate stratified layer --- i.e.\ the layer whose density matches the incoming mass most closely.
To model this effect, they use a \emph{control function} $B_c^i$ that selects a binary (0 or 1) value to multiply the water entry by at each layer,
$$ B_c^i = \left\{ \begin{array}{ll}
   1 & \text{if}\ T_{i-1} \geq T_c > T_i \\
   0 & \text{otherwise},
\end{array} \right. $$
where $T_c$ is the temperature of water incoming from the solar collector.
There is an analogous control function for cold water incoming from the load.
The model includes heat flow between nodes due to the net movement of water in the tank (accounting for water incoming at different nodes from the collector and load, and water being drawn off at the top and bottom of the tank), but not diffusion between layers.
It does not account for any energy added internal to the tank --- for example, heating elements.
This could be accounted for by introducing another mass in/outflow pair in the same fashion as the solar collector, which simply added a constant amount of energy to the water instead of an amount determined by the solar conditions.

\textcite{Pfeiffer11} describes a model of the stratified tank that does not include external hot water supply, but does include direct heat supply (i.e. an electric heating element).
The model describes the rate of change of temperature of each node as three separate factors:
$$ \flow{tot} = \flow{ext} + \flow{mix} + \flow{mflow}, $$
the changes in temperature due to external factors (heat loss to ambient, and a heating element), mixing (convection and conduction), and overall mass flow (i.e. the action of pumping water through the load circuit).
The mixing effect is a nonlinear function of several liquid constants and the derivative of the temperature gradient in the tank, given in \textcite{Hawlader88} as
$$ \epsilon_t = \left\{ \begin{array}{ll}
   (K \delta l)^2 \sqrt{g \beta \frac{\partial T}{\partial z}} & \text{if}\ \frac{\partial T}{\partial z} > 0 \\
   0 & \text{otherwise}.
\end{array} \right. $$

In contrast to the model described by Cristofari, the one used by Pfeiffer accounts explicitly for diffusion of heat by water buoyancy.
\authors{Cristofari02} make the assumption that slow effects such as diffusion will be insiginficant compared to the mass volume flows throughout the tank.

\section{Building automation}

\subsection{Prague}

A study of model-predictive control for heating, ventilation and air-conditioning (HVAC) systems was undertaken by \textcite{Siroky11} in the Czech Republic.
They implemented a controller for a building with three identical blocks, which provided a good opportunity to apply different control methods across near-uniform weather and lighting conditions.
Instead of cooling, this study was mainly concerned with heating the building during cold winder conditions.
Over their cross-comparisons between blocks with different control strategies, they found that the MPC controller saved between 15\% and 28\% of their energy bill.
(However, they concluded that any economic analysis of applying MPC control must also include the cost of model development, integration, and ongoing maintenance.)

In this application, the MPC controller specified setpoints for each room, and low-level controllers were responsible for ensuring these targets were met.
The authors note that they did not represent the setpoint targets as constraints in their optimisation problem, as this would often lead to infeasible optimisation problems.
Instead, they considered a cost function which significantly penalised deviations below the reference temperature, but was more tolerant to deviations above it.

This cost function also accounted for the energy bill, which was their desired minimisation target.
This required manually tweaking cost weighting matrices to ensure that the tradeoffs of deviating from the reference setpoints were balanced by the desire to minimise the building's energy bill.
They also considered the use of the $L_\infty$ norm on the input to reduce the \emph{peak} electricity usage.

The authors make several practical notes on the application of MPC\@.
They opine that stability is not often a concern in building control due to the slow and naturally stable dynamics.
They also emphasize the role that modelling building thermal mass plays in any form of HVAC control.
Finally, they note that a full building automation system could control not just the air conditioning, but lighting and blinds as well in search of an efficient environment.

\subsection{The OptiControl project}

The OptiControl is a project aimed at developing autonomous building control technology, based in Switzerland and funded collaboratively by industry and governments.
The project focuses on using MPC to control HVAC systems in buildings.
In its two-year progress report by \textcite{Gyalistras10}, a chapter is devoted to their findings related to MPC, authored by \textcite{Oldewurtel10}.

They note that MPC is a natural way of describing control problems, which allows designers to specify \emph{what} they want the overall system to achieve (i.e., the goal function), rather than \emph{how} to achieve the goal.
It is a high-level specification\footnotemark{} and lends itself to easy and even on-the-fly tuning --- for example, the authors note that the user may be allowed to set a preference between minimising energy usage and minimising their energy bill (which are sometimes mutually exclusive goals).
They also believe that a benefit of MPC is that an expert is not required to design control rules.

\footnotetext{Compare this to the commonly-cited benefit of functional programming languages over their imperative counterparts: the ease with which you can describe \emph{what} to compute rather than \emph{how} to about computing it.}

The authors describe in detail the control algorithms used in the OptiControl project, which focus on robustness in the face of disturbances.
Disturbances, in this case, are both model inaccuracies as well as inaccuracies in the weather prediction.
Though \authors{Siroky11} noted that stability is not usually a concern for building control, the OptiControl authors believe that robustness --- the guarantees a system makes in the face of uncertain disturbances --- is critical.
They examine three MPC formulations that deal with this uncertainty in different ways.

\begin{description}
   \item[Certainty equivalence]
      This is, according to Oldewurtel and colleagues, the most common MPC approach, and is the approach used by \authors{Siroky11}.
      This formulation of MPC assumes that the disturbances will exactly equal their mean (or expected) value during the execution of the plan.
      This is optimistic at best, and though the authors find it an unreasonable approach, they note that it can be made robust by tightening constraints to alleviate the effects of potential disturbances.

   \item[Chance constraints]
      Chance constraints formulate the constraints of an MPC problem as an inequality not on the state itself, but on the probability of the state exceeding its constraints, like so:
      $$ P(y_t \leq \bar{y}) \geq 1 - \alpha, $$
      where $\alpha$ is the desired probability that the constraints will be satisfied.
      This form is then used to derive appropriate bounds by which to shift $\bar{y}$ --- in effect, automating (or providing some mathematical basis for) the tightening described by the {\bf certainty equivalence} method.
      The authors found that this approach resulted in very conservative control, which they were unsatisfied with.

   \item[Affine disturbance feedback]
      The OptiControl project developed an MPC formulation that describes $u$ as a function of the prior disturbances to the system $w$, weighted by matrix $M$.
      They let
      $$ u_i = \sum _{j=0} ^{i-1} M_{ij} w_j + h_j $$
      and optimise over $\hat{M}$ instead of $\hat{u}$.
      They then insert this definition into the {\bf chance constraint} formulation and solve the resulting second-order cone problem.
      For performance reasons, they found it necessary to pre-compute various permutations of $M$ and combine them with linear weighting during the actual optimisation process.
\end{description}

The latter two methods were developed by and for the OptiControl project.

\section{Water tank control}

\subsection{University of California, Merced}

A study of a \emph{chilled} water tank took place on the campus of Berkeley university by \textcite{Ma12}.
The authors designed and implemented a controller for the university's chilling plant, which cools and stores water to distribute to the various air conditioners around campus.
This system did not directly control the air conditioning equipment; instead, its responsibility was to optimise the chilling process, usually at night before its use during the day.

Over four trials, they compared the hand-designed control regime the university had been operating under, a restricted MPC algorithm, an improved manual control regime informed by the improvements made by MPC, and finally an unrestricted MPC controller.
The unrestrained MPC controller performed best out of the four trials, increasing the power plant's efficiency by 19\% and reducing the daily HVAC electricity bill by 76\%.
They found that overall, the MPC strategy was very similar to the manual strategy used by the operators, but with more finely-tuned timing.

This study considered a stratified water tank where only two layers were modelled (a third was observed in actual data, but ignored).
The campus load was modelled as an RC network that accounted for wall and window thermal masses, thermal loads from occupants, lights, equipment, and environmental facturs such as ambient temperature and sunlight.
Weather prediction was used to predict the campus load based on historical data.

The controller included a robustness guarantee.
Historical disturbance data for the campus was used to construct a terminal constraint set on the MPC optimisation problem to ensure that the controller would chill enough water to guarantee the campus could be cooled if actual load exceeded predicted load.

In addition, the authors incorporated a binary (non-convex) control input --- the mass flow rate to the chillers --- by choosing to optimise not over a sequence of input values in time, but by restricting the controller to only use the chiller for one continuous period, and optimising over the start and end times of this period.
In this way they were able to avoid the difficulties in designing an optimal binary controller.
They became subject to the restriction of having only one chilling period per day, but the tradeoff seemed worth it.
Experienced human operators were unable to design a better control regime for the system, even after seeing and refining an initial solution by the predictive controller.

\subsection{Technical University of Denmark}
\label{sec:review:mpc:halvgaard}

\textcite{Halvgaard12} used their mixed-tank model (descibed in \autoref{sec:review:mixed-tank}) to perform what they term `economic MPC', by which they mean that their objective function measured the monetary cost to the operator of the water tank.
They used simulated load and weather predictions to evaluate the MPC controller they designed.
The load, importantly, involved three equal draw-offs at 7am, noon, and 7pm.
In addition, they used real wholesale electricity prices from the Elspot market, which has a significant off-peak discount in the early morning.

\authors{Halvgaard12} discovered savings of 25--30\% of the electricity bill with the MPC controlling the tank heating element during simulation.
This was achieved mainly by shifting the heating schedule to perform most of its work at night, when the spot price of electricy was cheaper.
The large tank may have contributed to these results against standard domestic consumption and solar production, allowing a night-time boost to provide adequate heat in the tank for an entire day of usage.
Smaller tanks may require more detailed controle regimes, for which the mixed-tank assumption may not prove adequate.

Over longer time horizons, the authors discovered that total power usage would increase even as the energy bill decreased, because the controller was considering only the energy bill, and consuming more cheap off-peak power.
The authors also noted that any prediction horizon longer than 24 hours did not noticeably affect the savings achieved, and nor did the accuracy of the weather forecasts.
