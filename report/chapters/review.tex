\chapter{Related and prior work}
\label{ch:review}

\section{Convex optimisation}

\todo{Astfalk, }

When applied to an MPC controller by \textcite{Wang10}, a bespoke quadratic program (QP) solver achieved a 200Hz control frequency on a system with 13 states and 3 inputs over a 30-step control horizon.
This was achievable by taking advantage of the structure of the MPC problem and the QP that was formed.
The authors implemented an algorithm that solved the problem with time complexity linear in the horizon dimension, rather than cubic as seen in interior-point methods.

\section{Model-predictive control}

\todo{Jalali, Wright, }

\section{Hot water system modelling}

\authors{Camacho04} estimate that one of the key differences in MPC control is the significant up-front modelling effort, as the system model is very important to the overall control achieved.
Therefore we take care to study various efforts in modelling hot water tanks.
In \autoref{ch:models} a model is developed following the work presented in this section, in order to evaluate the controllers designed in \autoref{ch:control}.
The model must be adequately able to capture the subtleties of the domestic hot water tank in order to better evaluate the effect of different control strategies.
Some of the details outlined here will also be used to derive a simplified tank model used in the controllers themselves, as control of the full nonlinear model is beyond the scope of this thesis.

\textcite{Vrettos12} note that many current publications use ``single-point'' or zero-dimensional models of domestic water heaters, where the entire tank is represented by a single value (temperature or energy).
We refer to these models as mixed-tank models, as they treat the entire tank as a single uniformly-mixed body of water.
The alternative is a one-dimensional model, in which the tank is considered to be at uniform conditions across every horizontal section, but may vary at different depths.
This type of model is able to capture the effect of stratification, described in \autoref{sec:review:stratified-tank-models}, which is an important factor in predicting the performance of hot water devices.

\subsection{Mixed-tank models}
\label{sec:review:mixed-tank}

\todo{DuffBradnum, Ayompe, Hollands?}

\textcite{Halvgaard12} made use of a mixed-tank model in order to perform economic model-predictive control of the heating elements.
Their model involves a single heat balance equation,
$$ C \dot{T} = Q_h + Q_s - Q_l - U A (T - T_a) $$
where
\begin{itemize}
   \item $C$ is the heat vapacity of water,
   \item $T$ is the temperature of the tank,
   \item $T_a$ is the ambient temperature,
   \item $Q_h$ is the temperature added to the tank by the electric heating element,
   \item $Q_s$ is the heat added by the solar collector,
   \item $Q_l$ is the heat removed by the load, and
   \item $U A$ is the product of heat transmission and surface area of the tank.
\end{itemize}
The heat gained from the element is a constrained input, the solar heat gain is a function of the instantaneous insolation, the load is based on predicted user activity and the loss is a simple physical equation based on the ambient temperature.
Their model was verified by measuring the temperature in a large water tank at eight different vertical locations and averaging these readings to calulate the approximate energy in the tank.
In a practical experiment they found that even with this simplified model of the tank, their controller was able to reduce the operating power bill by up to 30\%.
This is further discussed in \autoref{sec:review:mpc:halvgaard}.

\textcite{Cao14} also used a mixed-tank model when designing a large-scale solar water heating system for a hotel.
Their publication does not include a practical experiment, just simulation of the system design.

\subsection{Stratified models}
\label{sec:review:stratified-tank-models}

Stratification is the natural tendency of water to form layers of uniform temperature, between which there is relatively little heat flow.
\textcite{Hollands89} review the benefits of maintaining stratification in a hot water storage tank in their 1989 paper.
At this point, stratified tanks were just beginning to become the preferred paradigm for storage in domestic hot water systems.
Their review points out that systems that maintain tank stratification (usually by designing for low flow rates, but potentially by using good diffusers inside the tank) can improve performance by nearly 40\% in some circumstances.
\todo{Make sure I understand how. Also, what is `performance'?}

The authors review mathematical models for stratified tanks and note that it was usual to model the tank as several fixed-volume, variable-temperature layers.
\textcite{Kleinbach93} refer to this approach as the \emph{multinode} model, which they contrast to the \emph{plug flow} model, in which there are a variable number of layers, each with variable volume and temperature.
With as few as three of these \emph{multinode} layers, \authors{Hollands89} found that models could under-predict system performance by 10\%.
With complex supply and load patterns, they found that up to sixty-four layers were needed to capture the system's dynamics.

\textcite{Cristofari02} formulated a stratified tank model for simulation, rather than optimisation, of SWHSs in Corsica.
They assume that a perfect diffuser exists on the tank inputs and outputs, so that water added or removed is done so from an appropriate stratified layer --- i.e.\ the layer whose density matches the incoming mass most closely.
To model this effect, they use a \emph{control function} $B_c^i$ that selects a binary (0 or 1) value to multiply the water entry by at each layer,
$$ B_c^i = \left\{ \begin{array}{ll}
   1 & \text{if}\ T_{i-1} \geq T_c > T_i \\
   0 & \text{otherwise},
\end{array} \right. $$
where $T_c$ is the temperature of water incoming from the solar collector.
There is an analogous control function for cold water incoming from the load.
The model includes heat flow between nodes due to the net movement of water in the tank (accounting for water incoming at different nodes from the collector and load, and water being drawn off at the top and bottom of the tank), but not diffusion between layers.
It does not account for any energy added internal to the tank --- for example, heating elements.
This could be accounted for by introducing another mass in/outflow pair in the same fashion as the solar collector, which simply added a constant amount of energy to the water instead of an amount determined by the solar conditions.

\textcite{Pfeiffer11} describes a model of the stratified tank that does not include external hot water supply, but does include internal heat supply (i.e. electric heating elements).
This model also accounts for diffusion between adjacent nodes due to buoyancy, as well as due to the overall mass flow effect of the load.
The buoyancy effect is a function of several liquid constants and the derivative of the temperature gradient in the tank, given in \textcite{Hawlader88} as
$$ \epsilon_t = \left\{ \begin{array}{ll}
   (K \delta l)^2 \sqrt{g \beta \frac{\partial T}{\partial z}} & \text{if}\ \frac{\partial T}{\partial z} > 0 \\
   0 & \text{otherwise}.
\end{array} \right. $$

\section{Building automation}

\subsection{Prague}

A study of model-predictive control for heating, ventilation and
air-conditioning (HVAC) systems was undertaken by \textcite{Siroky11} in the Czech Republic.
They implemented a controller for a building with three identical blocks, which provided a good opportunity to apply different control methods across near-uniform weather and lighting conditions.
Instead of cooling, this study was mainly concerned with heating the building during cold winder conditions.
Over their cross-comparisons between blocks with different control strategies, they found that the MPC controller saved between 15\% and 28\% of their energy bill.
(However, they concluded that any economic analysis of applying MPC control must also include the cost of model development, integration, and ongoing maintenance.)

In this application, the MPC controller specified setpoints for each room, and low-level controllers were responsible for ensuring these targets were met.
The authors note that they did not represent the setpoint targets as constraints in their optimisation problem, as this would often lead to infeasible optimisation problems.
Instead, they considered a cost function which significantly penalised deviations below the reference temperature, but was more tolerant to deviations above it.

This cost function also accounted for the energy bill, which was their desired minimisation target.
This required manually tweaking cost weighting matrices to ensure that the tradeoffs of deviating from the reference setpoints were balanced by the desire to minimise the building's energy bill.
They also considered the use of the $L_\infty$ norm on the input to reduce the \emph{peak} electricity usage.

The authors make several practical notes on the application of MPC\@.
They opine that stability is not often a concern in building control due to the slow and naturally stable dynamics.
They also emphasize the role that modelling building thermal mass plays in any form of HVAC control.
Finally, they note that a full building automation system could control not just the air conditioning, but lighting and blinds as well in search of an efficient environment.

\subsection{The OptiControl project}

The OptiControl is a project aimed at developing autonomous building control technology, based in Switzerland and funded collaboratively by industry and governments.
The project focuses on using MPC to control HVAC systems in buildings.
In its two-year progress report by \textcite{Gyalistras10}, a chapter is devoted to their findings related to MPC, authored by \textcite{Oldewurtel10}.

They note that MPC is a natural way of describing control problems, which allows designers to specify \emph{what} they want the overall system to achieve (i.e., the goal function), rather than \emph{how} to achieve the goal.
It is a high-level specification\footnotemark{} and lends itself to easy and even on-the-fly tuning --- for example, the authors note that the user may be allowed to set a preference between minimising energy usage and minimising their energy bill (which are sometimes mutually exclusive goals).
They also believe that a benefit of MPC is that an expert is not required to design control rules.

\footnotetext{Compare this to the commonly-cited benefit of functional programming languages over their imperative counterparts: the ease with which you can describe \emph{what} to compute rather than \emph{how} to about computing it.}

The authors describe in detail the control algorithms used in the OptiControl project, which focus on robustness in the face of disturbances.
Disturbances, in this case, are both model inaccuracies as well as inaccuracies in the weather prediction.
Though \authors{Siroky11} noted that stability is not usually a concern for building control, the OptiControl authors believe that robustness --- the guarantees a system makes in the face of uncertain disturbances --- is critical.
They examine three MPC formulations that deal with this uncertainty in different ways.

\begin{description}
   \item[Certainty equivalence]
      This is, according to Oldewurtel and colleagues, the most common MPC approach, and is the approach used by \authors{Siroky11}.
      This formulation of MPC assumes that the disturbances will exactly equal their mean (or expected) value during the execution of the plan.
      This is optimistic at best, and though the authors find it an unreasonable approach, they note that it can be made robust by tightening constraints to alleviate the effects of potential disturbances.

   \item[Chance constraints]
      Chance constraints formulate the constraints of an MPC problem as an inequality not on the state itself, but on the probability of the state exceeding its constraints, like so:
      $$ P(y_t \leq \bar{y}) \geq 1 - \alpha, $$
      where $\alpha$ is the desired probability that the constraints will be satisfied.
      This form is then used to derive appropriate bounds by which to shift $\bar{y}$ --- in effect, automating (or providing some mathematical basis for) the tightening described by the {\bf certainty equivalence} method.
      The authors found that this approach resulted in very conservative control, which they were unsatisfied with.

   \item[Affine disturbance feedback]
      The OptiControl project developed an MPC formulation that describes $u$ as a function of the prior disturbances to the system $w$, weighted by matrix $M$.
      They let
      $$ u_i = \sum _{j=0} ^{i-1} M_{ij} w_j + h_j $$
      and optimise over $\hat{M}$ instead of $\hat{u}$.
      They then insert this definition into the {\bf chance constraint} formulation and solve the resulting second-order cone problem.
      For performance reasons, they found it necessary to pre-compute various permutations of $M$ and combine them with linear weighting during the actual optimisation process.
\end{description}

These latter two methods were developed by and for the OptiControl project.

\section{Water tank control}

\todo{Maybe: Azzouzi, Hasan, Yang, Sossan, AbdelMalek, Michaels}

\subsection{University of California, Merced}

A study of a \emph{chilled} water tank took place on the campus of Berkeley university by \textcite{Ma12}.
The authors designed and implemented a controller for the university's chilling plant, which cools and stores water to distribute to the various air conditioners around campus.
This system did not directly control the air conditioning equipment; instead, its responsibility was to optimise the chilling process, usually at night before its use during the day.

Over four trials, they compared the hand-designed control regime the university had been operating under, a restricted MPC algorithm, an improved manual control regime informed by the improvements made by MPC, and finally an unrestricted MPC controller.
The unrestrained MPC controller performed best out of the four trials, increasing the power plant's efficiency by 19\% and reducing the daily HVAC electricity bill by 76\%.
They found that overall, the MPC strategy was very similar to the manual strategy used by the operators, but with more finely-tuned timing.

This study considered a stratified water tank where only two layers were modelled (a third was observed in actual data, but ignored).
The campus load was modelled as an RC network that accounted for wall and window thermal masses, thermal loads from occupants, lights, equipment, and environmental facturs such as ambient temperature and sunlight.
Weather prediction was used to predict the campus load based on historical data.

The controller included a robustness guarantee.
Historical disturbance data for the campus was used to construct a terminal constraint set on the MPC optimisation problem to ensure that the controller would chill enough water to guarantee the campus could be cooled if actual load exceeded predicted load.

In addition, the authors incorporated a binary (non-convex) control input --- the mass flow rate to the chillers --- by choosing to optimise not over a sequence of input values in time, but by restricting the controller to only use the chiller for one continuous period, and optimising over the start and end times of this period.
In this way they were able to avoid the difficulties in designing an optimal binary controller.
\todo{Verify this is what they did. It sure looked like it, but\ldots}

\subsection{Technical University of Denmark}
\label{sec:review:mpc:halvgaard}

\textcite{Halvgaard12} used their mixed-tank model (descibed in \autoref{sec:review:mixed-tank}) to perform what they term `economic MPC', by which they mean that their objective function measured the monetary cost to the operator of the water tank.
They used simulated load and weather predictions to evaluate the MPC controller they designed.
The load, importantly, involved three equal draw-offs at 7am, noon, and 7pm.
In addition, they used real wholesale electricity prices from the Elspot market, which has a significant off-peak discount in the early morning.

\authors{Halvgaard12} discovered savings of 25--30\% of the electricity bill with the MPC controlling the tank heating element during simulation.
This was achieved mainly by shifting the heating schedule to perform most of its work at night, when the spot price of electricy was cheaper.
The large tank (over 700 litres) may have contributed to these results against standard domestic consumption and solar production, allowing a night-time boost to provide adequate heat in the tank for an entire day of usage.
Smaller tanks may require more detailed controle regimes, for which the mixed-tank assumption may not prove adequate.

Over longer time horizons, the authors discovered that total power usage would increase even as the energy bill decreased, because the controller was considering only the energy bill, and consuming more cheap off-peak power.
The authors also noted that any prediction horizon longer than 24 hours did not noticeably affect the savings achieved, and nor did the accuracy of the weather forecasts.
