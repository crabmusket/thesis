\chapter{System description}
\label{ch:system}

So far, several different system types have been encountered and several variables introduced that affect a solar domestic water service's modelling and operation.
I will now define concretely the system that will be analysed for the remainder of this thesis.
This includes qualitative and quantitative descriptions of the system dimensions and operation.

\todo{Don't forget to refer to \autoref{sec:background:system}}

\section{Hardware}

The hot water service considered by this thesis outlined in \autoref{fig:system-overview} and contains the following elements:
\begin{itemize}
   \item A 350L hot water tank containing stratification-enhancing manifolds attached to all inlets.
   \item An 800mL solar collector which spans an area of 5 square metres.
   \item A 200mL auxiliary heating circuit with the equivalent of a 3kW heating element.
   \item Two pumps, one to force water through the solar collector, and one to force water through the auxiliary heating loop. Pumps are represented in the figure as circles with a triangle inside indicating their only available direction.
         The power rating of these pumps is not considered, and the energy they consume is ignored during simulations.
   \item A three-way valve, denoted in the figure as a circle containing crossed lines. It receives input from the left and may direct it either to the right, or downwards.
         The valve is assumed to be electronically or mechanically controlled by a thermostat (see below).
   \item Not shown on the diagram are thermometers at the top of the tank, and inside the tops of both the collector and auxiliary heating loops.
         Three thermostats are connected between these thermometers and the pumps and valve.
         The operation of these thermostats is detailed in \autoref{sec:system:control}.
\end{itemize}

\begin{figure}
   \centering
   \begin{lpic}{images/system-overview(7.5cm,)}
      \lbl{97.5,57; \LaTeX}
   \end{lpic}
   \caption{The solar hot water service under consideration}
   \label{fig:system-overview}
\end{figure}
\todo{label the diagram}

Note that there are several omissions in this simplified model.
A practical system would include pressure release valves in both the solar collector and the heated tank itself to allow venting of excess heat on hot days.
This is a normal part of direct heating system operation.
There is no plumbing considered between the tank and the collector, or between the tank and the load --- it is assumed that these transfers happen over zero distance.

\section{Internal control}
\label{sec:system:control}

The hot water service has three actuators and two sensors, described above.
They are used by three internal thermostat controllers.

\begin{description}
   \item[Collector valve controller]

      The collector valve thermostat measures the difference in temperature at the top of the collector loop and at the bottom of the storage tank.
      If the water in the collector is more than $8\degr$ warmer than the bottom of the tank, the collector valve will send water into the tank.
      If the water is not hot enough, it will be sent downwards to proceed through the collector loop again.

      The thermostat includes a $6\degr$ hysterisis band, so the collector will keep pumping until the temperature difference at the bottom of the tank is $2\degr$.
      Note that since the temperature at the bottom of the tank is measured, water may enter the tank that is colder than the top of the tank.
      This is taken care of by the stratification manifold attached to the inlet.

      The auxiliary pump is assumed to be always active in order to prevent stagnation in the collector.
      Its effect on the tank is decided by whether water from the collector returns to the tank, or to the collector.

   \item[Auxiliary pump controller]

      The auxiliary pump is not always on, but is controlled by an absolute thermostat.
      If the temperature at the top of the tank falls below $50\degr$ then the auxiliary pump will turn on and send water through the auxiliary loop.
      When the temperature at the top of the tank climbs to $55\degr$ the auxiliary pump will switch off.

   \item[Auxiliary heat controller]

      Control of the auxiliary heater is typically separated from control of the pump.
      The heater thermostat measures the temperature at the top of the auxiliary loop, before the water is returned to the tank.
      If the water is cooler than $50\degr$ \emph{and} the auxiliary pump is currently on, the auxiliary heater will engage until the water is at $55\degr$.
      It is designed to ensure that water is never heated when stagnant.
\end{description}

\section{Site and situation}

Describe physical environment/location/other relevant factors exterior to the system itself, including probable load conditions, ambient temperature, etc.
\todo{write this}

\begin{itemize}
   \item Location of the system -- therefore likely weather patterns.
   \item Load conditions (i.e. how many people in the house, what other equipment).
\end{itemize}
