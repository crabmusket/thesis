\chapter{Models}
\label{ch:models}

\section{Simulation models}

A hot water tank is a complex nonlinear system when stratification is considered.
It is desirable to develop an accurate simulation of the system in order to test the validity of controllers designed with simplified models.

A popular tank model, as discussed in \autoref{sec:review:stratified-tank-models}, treats the tank as a stack of identical vertical disks with their own heat flow equations.
I will reproduce this model, relying on the description in \textcite{Pfeiffer11} of \textcite{Koch11}, which this author has not been able to base work off first-hand.
\todo{Is that ok?}
The solar hot water input will be handled in a similar fashion to that of \textcite{Cristofari02}, who also used a discretised hot water tank, but had no electric heaters, only dealing with flows of hot and cold water into and out of the tank.
Note that this approach assumes stratification baffle devices are present on the solar and mains inputs into the tank, and that these devices work perfectly.

\subsection{Discretised tank model}

The tank is split into $N$ identical horizontal disks as illustrated in \autoref{fig:models:discretised-tank}.
One of these disks is detailed in \autoref{fig:models:discretised-disk}.
Each element is subjected to several heat flows that affect its temperature:
\begin{equation}
   \label{eq:flow-total}
   \flow{tot} = \flow{ext} + \flow{mix} + \flow{mflow},
\end{equation}
where $\flow{tot}$ is the final rate of change of the temperature of disk $i$.

The heat flows to and from the exterior caused by the heating element and losses to the tank wall, given by $\flow{ext}$, are
\begin{eqnarray}
   \label{eq:flow-ext}
   \flow{ext} &=& \flow{el} - \flow{loss}, \nonumber\\
              &=& \left(\frac{P_{\text{el}}}{m_i c n_{\text{el}}}\right) u
                + \left(\frac{A_{\text{wall}, i} U}{m_i c}\right) (T_i - T_{\text{amb}}).
\end{eqnarray}

The heat exchange through natural buoyancy and diffusion are
\begin{eqnarray}
   \label{eq:flow-mix}
   \flow{mix} &=& \flow[i+1]{out} + \flow[i-1]{out} - 2 \flow{out}, \\
   \label{eq:flow-mix-out}
   \flow{out} &=& \left(\frac{k}{\rho c d^2} + \epsilon_{t, i}\right)T_i
\end{eqnarray}
Note special cases for nodes 0 and $N-1$ in the implementation (code listing in \autoref{app:code:models:tank}).
Buoyancy factor:
\begin{equation}
   \label{eq:epsilon}
   \epsilon_{t, i} = (K \delta l)^2 \sqrt{g \beta \max \left\{ 0, \frac{\partial T}{\partial z}\right\}}
\end{equation}
Explain $\frac{\partial T}{\partial z}$.

Finally, the heat flow caused by water being forced through the tank is given by
\begin{eqnarray}
   \label{eq:flow-mflow}
   \flow{mflow} &=& \frac{\dot{m}}{m_i} (T_{i-1} - T_i).
\end{eqnarray}

\subsection{Solar hot water input}

\subsection{Model validity}

\section{Controller models}

\subsection{Discrete-time state space model}

This is how we make a state-space model, consider the system as linear time-varying, but say it in a way that doesn't exclude the possibility that I might come up with other models, for example based on step response.
\todo{Fix these words.}
\begin{eqnarray}
   \label{eq:continuous-xdot}
   \dvec{x} &=& A(t) \vec{x} + B(t) \vec{u}, \\
   \label{eq:continuous-y}
   \vec{y} &=& C(t) \vec{x}
\end{eqnarray}
where $\vec{x}$ represents the state vector, $\vec{u}$ the input vector, $\vec{y}$ the output vector, and $t$ the current time.
From now, all $t$ arguments will be dropped for clarity; is it understood that all matrices may vary in time, and that $\dvec{x}$ and $\vec{y}$ are functions of time.
We discretise this system in the usual manner for linear state-space systems, assuming a zero-order hold on input signals between each time step, forming a new set of discrete-time equations
\begin{eqnarray}
   \label{eq:discretise-A}
   A_d &=& e^{A \delta t}, \\
   \label{eq:discretise-B}
   B_d &=& A^{-1} (A_d - I) B, \\
   \label{eq:discrete-xdot}
   \vec{x}_{t + \delta t} &=& A_d \vec{x} + B_d \vec{u}, \\
\end{eqnarray}
though \autoref{eq:continuous-y} is unchanged.

\subsection{Treating disturbances}

In this simple model, we will include an explicit handling of `disturbance' inputs.
These refer to any system inputs which we do not have direct control over.
In the case of a water heating tank, disturbances include heat loss to the environment, heat contribution from solar collector, and the user load schedule.
Our only non-disturbance input is, of course, the electric heating element's state.

To make this distinction between disturbance and non-disturbance inputs, we decompose $B$ and $\vec{u}$ into
\begin{eqnarray}
   B &=& \left[\begin{array}{cc}
      B_u & B_d
   \end{array}\right], \\
   \vec{u} &=& \left[\begin{array}{c}
      \vec{u}_e \\ \vec{u}_d
   \end{array}\right]
\end{eqnarray}
where $\vec{u}_e$ represents our \emph{explicit} control signal to the system.
When $B$ is discretised according to \autoref{eq:discretise-B}, we can simply reconstruct the discretised $B_{u, d}$ and $B_{d, d}$ from the appropriate columns of $B_d$.

\subsection{Control problem}

\todo{Diagram of $\hvec{y}, \hvec{u}$ over planning horizon.}
MPC controllers solve an optimisation problem at every instant in time to determine the next control input.
The form of this optimisation problem is, in general,
\begin{equation}
   \label{eq:mpc-opt}
   \optimise
      {minimise}{\hvec{u}}
      {f(\hvec{y}, \hvec{u})}
      {\hvec{y} = \Psi \vec{x}_0 + \Theta \hvec{u}}
\end{equation}
The controller attempts to minimise some objective function $f$ which usually processes the vector of outputs $\hvec{y}$ and inputs $\hvec{u}$ over the time horizon.
This time horizon, $H$, is a unitless integer that determines how many time steps into the future the planning problem will consider.
We first define the output, input and disturbance vectors used in the planning problem, in terms of the time horizon:
\begin{eqnarray}
   \label{eq:mpc-vectors}
   \hvec{y} = \left[\begin{array}{c}
      \vec{y}_0 \\
      \vdots \\
      \vec{y}_H
   \end{array}\right],
   &
   \hvec{u} = \left[\begin{array}{c}
      \vec{u}_0 \\
      \vdots \\
      \vec{u}_H
   \end{array}\right],
   & \text{and} \quad
   \hvec{d} = \left[\begin{array}{c}
      \vec{d}_0 \\
      \vdots \\
      \vec{d}_H
   \end{array}\right].
\end{eqnarray}
We can now define the matrices $\Psi$ and $\Theta$:
\begin{eqnarray}
   \label{eq:mpc-psi}
   \Psi &=& \left[\begin{array}{c}
      CA \\ CA^2 \\ \vdots \\ CA^H
   \end{array}\right]
   \\\label{eq:mpc-theta-u}
   \Theta_u &=& \left[\begin{array}{cccc}
      CB_u & 0 & 0 & 0 \\
      CAB_u & \ddots & 0 & 0 \\
      \vdots & \ddots & \ddots & 0 \\
      CA^{H-1}B_u & \cdots & CAB_u & CB_u
   \end{array}\right]
   \\\label{eq:mpc-theta-d}
   \Theta_d &=& \left[\begin{array}{cccc}
      CB_d & 0 & 0 & 0 \\
      CAB_d & \ddots & 0 & 0 \\
      \vdots & \ddots & \ddots & 0 \\
      CA^{H-1}B_d & \cdots & CAB_d & CB_d
   \end{array}\right]
   \\\label{eq:mpc-theta}
   \Theta &=& \left[\begin{array}{cc}
      \Theta_u & \Theta_d
   \end{array}\right]
\end{eqnarray}

\subsection{Halvgaard mixed-tank model}
\label{sec:model:halvgaard}

\subsection{Stratified tank model}

Model the water tank as a perfect cylinder of height $h$ and radius $r$.
Divide the tank into $N$ vertical slices, where slice 1 is at the top and slice $N$ is at the bottom, with an equal fixed height (and therefore fixed volume $V = \pi r^2 \frac{h}{N}$).
Define $T_1 \dots T_N$ as the temperatures of each slice.
Define $\dot{m}_c$ and $\dot{m}_l$ as the mass flows through the collector and load circuits respectively.
The mass flows are `positive clockwise' as viewed on the diagram (collector flow enters at top, load flow at bottom).
Define $\dot{m}_{m, i}$ as the mass flow through each node (positive downwards) which takes into account the mass flow through node $i$ given the behavior of the collector and load functions.
$$ \dot{m}_{m, i} = \dot{m}_c \sum _{j=1} ^{i-1} B^j_c - \dot{m}_l \sum _{j=i+1} ^N B^j_l $$
Therefore the head flow in a node $i$ is:
$$ \dot{Q} = m C \dot{T}_i $$
