\chapter{Models}
\label{ch:models}

\section{Simulation models}

A hot water tank is a complex nonlinear system when stratification is considered.
It is desirable to develop an accurate simulation of the system in order to test the validity of controllers designed with simplified models.

The popular multinode tank model, as discussed in \autoref{sec:review:stratified-tank-models}, treats the tank as a stack of identical vertical disks with their own heat flow equations.
I will reproduce this model, relying on the description in \textcite{Cristofari02} with my own augmentation to handle the auxiliary heater, which was not present in that work.
Note that this approach assumes stratification baffle devices are present on the solar and mains inputs into the tank, and that these devices work perfectly.

\subsection{Discretised tank model}

The tank is split into $N$ identical horizontal disks as illustrated in \autoref{fig:models:discretised-tank}, numbered $0$ (bottom) to $N-1$ (top).
One of these disks is detailed in \autoref{fig:models:discretised-disk}.
The heat flows on this diagram are governed by the equation:
\begin{equation}
   \label{eq:node-dT}
   \rho_i C_i v_i \dot{T}_i = \Q{amb} + \Q{mflow} + \Q{inlet}
\end{equation}
That is, the total energy flow through the node can be broken into three parts: ambient losses, changes from the mass flow through the tank while charging/drawing, and input from the three heat/cold inlet: the collector inlet, mains inlet, and auxiliary inlet.

The ambient term is a simple equation accounting for the surface area of an element and its current temperature relative to the exterior ambient temperature.
The parameter $U_{s, i}$ accounts for the different surface areas of the nodes at each end of the tank.
\begin{equation}
   \Q{amb} = U_{s, i} (T_a - T_i)
\end{equation}
The mflow energy flow accounts for the heat that enters and exits each node carried by the actual volume of water that flows through it.
As water is physically flowing through the tank during charging (hot water entering from the solar collector) and drawing (hot water exiting to the load), the composition of each node changes.
The factor $\dot{m}_i$, defined in \autoref{eq:mdot}, refers to the amount of water entering node $i$ from node $i+1$, the node above it, and is therefore positive when water is flowing downwards through the tank (in the direction of drawing).
\begin{equation}
   \Q{mflow} = \max \left\{ 0, \dot{m}_i \right\}     C_i (T_{i+1} - T_i)
             + \min \left\{ 0, \dot{m}_{i-1} \right\} C_i (T_i - T_{i-1})
\end{equation}
We additionally define $\dot{m}_{N-1} = \dot{m}_{-1} = 0$ to handle the end cases.
Finally, the heat flow from each inlet is calculated in terms of its current mass flow and temperature.
$\Q{aux}$ represents the head gained due to the auxiliary input, which will be covered below.
\begin{equation}
   \label{eq:Q-inlet}
   \Q{inlet} = \B{col} \dot{m}_c (T_c - T_i)
             + \B{load} \dot{m}_l (T_l - T_i)
             + \Q{aux}
\end{equation}
where $\Q{aux}$ is the heat from the auxiliary input, defined below in \autoref{eq:Q-aux}.

\Autoref{eq:Q-inlet} introduced the two \emph{control functions} $\B{col}$ and $\B{load}$ used by \authors{Cristofari02}, which are binary indicator functions that determine which nodes receive hot and cold water flows from the inlets.
These control functions simulate the hot and cold water from the collector and load passing through stratification devices as they enter the tank, so that they are distributed to the water layer which most appropriately matches their temperature.
They only ever return 1 for a single node in the tank.
They are defined as follows:
\begin{eqnarray}
   \B{col} &=& \begin{dcases*}
      1 & if $T_c > T_i$ and $i = N-1$ \\
      1 & if $T_{i+1} \ge T_c > T_i$ and $i < N-1$ \\
      0 & otherwise
   \end{dcases*}
   \\
   \B{load} &=& \begin{dcases*}
      1 & if $T_l < T_i$ and $i = 0$ \\
      1 & if $T_{i-1} \le T_l < T_i$ and $i > 0$ \\
      0 & otherwise
   \end{dcases*}
\end{eqnarray}
The mass flow through each node is defined by \authors{Cristofari02} using sums of these collector functions to express the presence of an inlet above/below the current node being examined as
\begin{equation}
	\label{eq:mdot}
   \dot{m}_i = \dot{m}_l \sum_{j=1}^{i-1} \B{load}
             - \dot{m}_c \sum_{j=i+1}^{N-1} \B{col}
             + \dot{m}_{\text{aux}, i}.
\end{equation}
$\dot{m}_{\text{aux}, i}$ represents the mass flow due to the auxiliary heating loop described in \autoref{eq:mdot-aux}.
Note again that the flow is positive in the direction of the load loop (downwards).

\subsection{Auxiliary input}

In addition to the solar collector providing hot water to the tank, we wish to simulate an auxiliary heater, which in an actual system may be a gas burner, or an electric element connected to mains power.
For the purposes of our simulation we will treat this auxiliary heat as a second collector loop which draws water from the node $i = O$ (for outlet), heats it directly, and re-introduces it into the tank using the usual stratification mechanism.
The heating element has a fixed power rating $P$, but we are able to control the mass flow through the auxiliary water loop, $\dot{m}_x$.
The temperature of the water re-entering the tank after being heated is denoted by $T_x$.
It is calculated by assuming that the water drawn out of node $O$, $T_O$, is constant across the time interval we simulate over, and calculating the temperature gain given the heater's power rating:
\begin{equation}
	\label{eq:T-aux}
	T_x = T_O + \frac{P}{m_x * C_i}.
\end{equation}
Of course, this equation is only defined for $m_x \ne 0$.
In cases of zero flow through the auxiliary loop, this calculation is not performed.
The heat entering each node due to the auxiliary heater is given by
\begin{equation}
   \label{eq:Q-aux}
	\Q{aux} = \B{aux} \dot{m}_x (T_x - T_i),
\end{equation}
where $\B{aux}$, the auxiliary inlet's control function, is defined similarly to that of the collector (since both are treated as `hot' water):
\begin{equation}
   \label{eq:B-aux}
   \B{aux} = \begin{dcases*}
		0 & if $i < O$ \\
      1 & if $T_x > T_i$ and $i = N-1$ \\
      1 & if $T_{i+1} \ge T_x > T_i$ and $i < N-1$ \\
      0 & otherwise.
   \end{dcases*}
\end{equation}
Finally, we must define the mass flow through a node due to the operation of the auxiliary loop.
It is slightly more complex than the mass flows caused by the collector and load loops, because their outlets are fixed at the bottom and top of the tank respectively.
It is given by
\begin{equation}
   \label{eq:mdot-aux}
	\dot{m}_{\text{aux}, i} = \dot{m}_l \left( o_i - \sum_{j=1}^{i-1} \B{aux} \right)
\end{equation}
where $o_i$ is a binary factor that selects nodes above the auxiliary outlet:
$$
	o_i = \begin{dcases*}
		1 & if $i \ge O$ \\
		0 & otherwise.
   \end{dcases*}
$$

The Python source code implementing this tank model is found in \autoref{app:code:models:tank2}.

\subsection{Model validity}

My implementation of the model was not verified against empirical data.
As this work focuses on comparing control strategies on the system, efforts were not made to ensure that physical parameters were realistic, only plausible.

\section{Weather and load pattern models}


\section{Controller models}

\subsection{Linear state space model}

This is how we make a state-space model, consider the system as linear time-varying, but say it in a way that doesn't exclude the possibility that I might come up with other models, for example based on step response.
\todo{Fix these words.}
\begin{eqnarray}
   \label{eq:continuous-xdot}
   \dvec{x} &=& A(t) \vec{x} + B(t) \vec{u}, \\
   \label{eq:continuous-y}
   \vec{y} &=& C(t) \vec{x}
\end{eqnarray}
where $\vec{x}$ represents the state vector, $\vec{u}$ the input vector, $\vec{y}$ the output vector, and $t$ the current time.
From now, all $t$ arguments will be dropped for clarity; it is understood that all matrices may vary in time, and that $\dvec{x}$ and $\vec{y}$ are functions of time.
We discretise this system in the usual manner for linear state-space systems, assuming a zero-order hold on input signals between each time step, forming a new set of discrete-time equations
\begin{eqnarray}
   \label{eq:discretise-A}
   A_d &=& e^{A \delta t}, \\
   \label{eq:discretise-B}
   B_d &=& A^{-1} (A_d - I) B, \\
   \label{eq:discrete-xdot}
   \vec{x}_{t + \delta t} &=& A_d \vec{x} + B_d \vec{u}, \\
\end{eqnarray}
though \autoref{eq:continuous-y} is unchanged.

\subsection{Treating disturbances}

In this simple model, we will include an explicit handling of `disturbance' inputs.
These refer to any system inputs which we do not have direct control over.
In the case of a water heating tank, disturbances include heat loss to the environment, heat contribution from solar collector, and the user load schedule.
Our only non-disturbance input is, of course, the electric heating element's state.

To make this distinction between disturbance and non-disturbance inputs, we decompose $B$ and $\vec{u}$ into
\begin{eqnarray}
   B &=& \left[\begin{array}{cc}
      B_u & B_d
   \end{array}\right], \\
   \vec{u} &=& \left[\begin{array}{c}
      \vec{u}_e \\ \vec{u}_d
   \end{array}\right]
\end{eqnarray}
where $\vec{u}_e$ represents our \emph{explicit} control signal to the system.
When $B$ is discretised according to \autoref{eq:discretise-B}, we can simply reconstruct the discretised $B_{u, d}$ and $B_{d, d}$ from the appropriate columns of $B_d$.

\subsection{Control problem}

\todo{Diagram of $\hvec{y}, \hvec{u}$ over planning horizon.}
MPC controllers solve an optimisation problem at every instant in time to determine the next control input.
The form of this optimisation problem is, in general,
\begin{equation}
   \label{eq:mpc-opt}
   \optimise
      {minimise}{\hvec{u}}
      {f(\hvec{y}, \hvec{u})}
      {\hvec{y} = \Psi \vec{x}_0 + \Theta \hvec{u}}
\end{equation}
The controller attempts to minimise some objective function $f$ which usually processes the vector of outputs $\hvec{y}$ and inputs $\hvec{u}$ over the time horizon.
This time horizon, $H$, is a unitless integer that determines how many time steps into the future the planning problem will consider.
We first define the output, input and disturbance vectors used in the planning problem, in terms of the time horizon:
\begin{eqnarray}
   \label{eq:mpc-vectors}
   \hvec{y} = \left[\begin{array}{c}
      \vec{y}_0 \\
      \vdots \\
      \vec{y}_H
   \end{array}\right],
   &
   \hvec{u} = \left[\begin{array}{c}
      \vec{u}_0 \\
      \vdots \\
      \vec{u}_H
   \end{array}\right],
   & \text{and} \quad
   \hvec{d} = \left[\begin{array}{c}
      \vec{d}_0 \\
      \vdots \\
      \vec{d}_H
   \end{array}\right].
\end{eqnarray}
We can now define the matrices $\Psi$ and $\Theta$:
\begin{eqnarray}
   \label{eq:mpc-psi}
   \Psi &=& \left[\begin{array}{c}
      CA \\ CA^2 \\ \vdots \\ CA^H
   \end{array}\right]
   \\\label{eq:mpc-theta-u}
   \Theta_u &=& \left[\begin{array}{cccc}
      CB_u & 0 & 0 & 0 \\
      CAB_u & \ddots & 0 & 0 \\
      \vdots & \ddots & \ddots & 0 \\
      CA^{H-1}B_u & \cdots & CAB_u & CB_u
   \end{array}\right]
   \\\label{eq:mpc-theta-d}
   \Theta_d &=& \left[\begin{array}{cccc}
      CB_d & 0 & 0 & 0 \\
      CAB_d & \ddots & 0 & 0 \\
      \vdots & \ddots & \ddots & 0 \\
      CA^{H-1}B_d & \cdots & CAB_d & CB_d
   \end{array}\right]
   \\\label{eq:mpc-theta}
   \Theta &=& \left[\begin{array}{cc}
      \Theta_u & \Theta_d
   \end{array}\right]
\end{eqnarray}

\subsection{Halvgaard mixed-tank model}
\label{sec:model:halvgaard}

\subsection{Stratified tank model}

Model the water tank as a perfect cylinder of height $h$ and radius $r$.
Divide the tank into $N$ vertical slices, where slice 1 is at the top and slice $N$ is at the bottom, with an equal fixed height (and therefore fixed volume $V = \pi r^2 \frac{h}{N}$).
Define $T_1 \dots T_N$ as the temperatures of each slice.
Define $\dot{m}_c$ and $\dot{m}_l$ as the mass flows through the collector and load circuits respectively.
The mass flows are `positive clockwise' as viewed on the diagram (collector flow enters at top, load flow at bottom).
Define $\dot{m}_{m, i}$ as the mass flow through each node (positive downwards) which takes into account the mass flow through node $i$ given the behavior of the collector and load functions.
$$ \dot{m}_{m, i} = \dot{m}_c \sum _{j=1} ^{i-1} B^j_c - \dot{m}_l \sum _{j=i+1} ^N B^j_l $$
Therefore the head flow in a node $i$ is:
$$ \dot{Q} = m C \dot{T}_i $$
