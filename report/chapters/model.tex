\chapter{Models}
\label{ch:models}

This chapter introduces the various models used in both simulation of the system, and control.
\Autoref{sec:models:simulation} introduces the full nonlinear tank model which is used when running simulations.
\Autoref{sec:models:control} outlines the linear state-space model of the stratified tank for MPC, and then describes the simplifying assumptions made to express the nonlinear simulation model as a linear model.
These sections all use the variables defined in \autoref{tab:model-symbols}.

\begin{table}
   \caption{Symbols used in model equations}
   \label{tab:model-symbols}
   \begin{center}
   \begin{tabular}{l l}
      \toprule
      $N_T$ & Number of nodes in the discretised tank model \\
      $N_C$ & Number of nodes in the discretised collector loop model \\
      $N_X$ & Number of nodes in the discretised auxiliary loop model \\
      $T_i$ & Temperature of node $i$ in the discretised model \\
      $T_c$ & Temperature of hot water from the collector entering the tank \\
      $T_x$ & Temperature of hot water from the auxiliary heater entering the tank \\
      $T_l$ & Temperature of the mains water entering the tank \\
      $T_a$ & Temperature of the air outside the tank \\
      $T_T$ & Target temperature for hot water inlets \\
      $\dot{m}_l$ & Rate at which water is flowing through the load loop \\
      $\dot{m}_c$ & Rate at which water is flowing through the collector loop\\
      $\dot{m}_x$ & Rate at which water is flowing through the auxiliary loop\\
      $\Q{*}$ & Energy inflow into tank node $i$ caused by * \\
      $\B{*}$ & Control/indicator function for flow * at node $i$ \\
      $\rho_w$ & Density of water \\
      $C$ & Thermal capacitance of water \\
      $v_T$ & Volume of a node in the tank \\
      $v_C$ & Volume of a node in the collector loop \\
      $v_X$ & Volume of a node in the auxiliary loop \\
      $U_s$ & Coefficient of surface temperature loss \\
      $\rho$ & Controller tuning parameter \\
      \bottomrule
   \end{tabular}
   \end{center}
\end{table}

\section{System description}
\label{sec:model:system}

So far, several different system types have been encountered and several variables introduced that affect a solar domestic water service's modelling and operation.
I will now define concretely the system that will be analysed for the remainder of this thesis.
Refer to \autoref{sec:background:system} for background information.

\subsection{Hardware}

The hot water service considered by this thesis outlined in \autoref{fig:system-overview} and contains the following elements:
\begin{itemize}
   \item A 350L hot water tank containing stratification-enhancing manifolds attached to all inlets.
   \item An 800mL solar collector which spans an area of 5 square metres.
   \item A 200mL auxiliary heating circuit with the equivalent of a 3kW heating element.
   \item Two pumps, one to force water through the solar collector, and one to force water through the auxiliary heating loop. Pumps are represented in the figure as circles with a triangle inside indicating their only available direction.
         The power rating of these pumps is not considered, and the energy they consume is ignored during simulations.
   \item A three-way valve, denoted in the figure as a circle containing crossed lines. It receives input from the left and may direct it either to the right, or downwards.
         The valve is assumed to be electronically or mechanically controlled by a thermostat (see below).
   \item Not shown on the diagram are thermometers at the top of the tank, and inside the tops of both the collector and auxiliary heating loops.
         Three thermostats are connected between these thermometers and the pumps and valve.
         The operation of these thermostats is detailed in \autoref{sec:models:system:control}.
\end{itemize}

\begin{figure}
   \centering
   \begin{lpic}{images/system-overview(6.5cm,)}
   \end{lpic}
   \caption{The solar hot water service under consideration}
   \label{fig:system-overview}
\end{figure}

Note that there are several omissions in this simplified model.
A practical system would include pressure release valves in both the solar collector and the heated tank itself to allow venting of excess heat on hot days.
This is a normal part of direct heating system operation.
There is no plumbing considered between the tank and the collector, or between the tank and the load --- it is assumed that these transfers happen over zero distance.

\subsection{Internal control}
\label{sec:models:system:control}

The hot water service has three actuators and two sensors, described above.
They are used by three internal thermostat controllers.

\begin{description}
   \item[Collector valve controller]
      The collector valve thermostat measures the difference in temperature at the top of the collector loop and at the bottom of the storage tank.
      If the water in the collector is more than $8\degr$ warmer than the bottom of the tank, the collector valve will send water into the tank.
      If the water is not hot enough, it will be sent downwards to proceed through the collector loop again.

      The thermostat includes a $6\degr$ hysterisis band, so the collector will keep pumping until the temperature difference at the bottom of the tank is $2\degr$.
      Note that since the temperature at the bottom of the tank is measured, water may enter the tank that is colder than the top of the tank.
      This is taken care of by the stratification manifold attached to the inlet.

      The auxiliary pump is assumed to be always active in order to prevent stagnation in the collector.
      Its effect on the tank is decided by whether water from the collector returns to the tank, or to the collector.

   \item[Auxiliary pump controller]
      The auxiliary pump is not always on, but is controlled by an absolute thermostat.
      If the temperature at the top of the tank falls below $50\degr$ then the auxiliary pump will turn on and send water through the auxiliary loop.
      When the temperature at the top of the tank climbs to $55\degr$ the auxiliary pump will switch off.

   \item[Auxiliary heat controller]
      Control of the auxiliary heater is typically separated from control of the pump.
      The heater thermostat measures the temperature at the top of the auxiliary loop, before the water is returned to the tank.
      If the water is cooler than $50\degr$ \emph{and} the auxiliary pump is currently on, the auxiliary heater will engage until the water is at $55\degr$.
      It is designed to ensure that water is never heated when stagnant.
\end{description}

\subsection{Site and situation}

The solar system is taken to be located in Sydney, Australia, and weather data for that city is used in simulations.
Local obstructions to solar radiation such as nearby buildings or trees are not considered, so the collector is assumed to be mounted in a sufficiently high place, or far away from such obstructions.

The load profile will be taken from standard datasets (described in \autoref{sec:evaluation:schedules}), and is taken to represent the median profile for a four-person family.
It is assumed that all members of the family are absent during the day, and take showers in the morning.
Light usage at night is accounted for.

\section{Simulation model}
\label{sec:models:simulation}

A hot water tank is a complex nonlinear system when stratification is considered.
It is desirable to develop an accurate simulation of the system in order to test the validity of controllers designed with simplified models.

The popular multinode tank model, as discussed in \autoref{sec:review:stratified-tank-models}, treats the tank as a stack of identical vertical disks with their own heat flow equations.
I will reproduce this model, relying on the description in \textcite{Cristofari02} with my own augmentation to handle the auxiliary heater, which was not present in that work.
Note that this approach assumes stratification baffle devices are present on the solar and mains inputs into the tank, and that these devices work perfectly.

\subsection{Discretisation}

The tank is split into $N_T$ identical horizontal disks as illustrated in \autoref{fig:discretised-tank}, numbered $0$ (bottom) to $N_T-1$ (top).
Each disk, called a node, has uniform volume $v_T$.
The heat flows into the node are governed by the equation:
\begin{equation}
   \label{eq:tank-node-dT}
   \rho_w C v_T \dot{T}_i = \Q{amb} + \Q{$T$, mflow} + \Q{inlet}.
\end{equation}
That is, the total energy flow through the node can be considered in three distinct parts: ambient losses, changes due to the mass flow through the tank while charging/drawing, and input from the three heat/cold inlets: the collector inlet, mains inlet, and auxiliary inlet.

\begin{figure}
   \centering
   \begin{lpic}{images/nodes(4.5cm,)}
      \lbl{31,49; $N_T$}
      \lbl{-5,48; $N_C$}
      \lbl{65,62; $N_X$}
      \lbl{31,64; $\dot{m}_l$}
      \lbl{20,64; $\dot{m}_c$}
      \lbl{42,64; $\dot{m}_x$}
   \end{lpic}
   \caption{The discretised tank, collector, and booster}
   \label{fig:discretised-tank}
\end{figure}

The collector and auxiliary heater are also treated as discretised volumes of water with $N_C$ and $N_X$ nodes of volumes $v_C$ and $v_X$ respectively.
The indices $i$ of the collector nodes begin at $N_T$ and end at $N_T+N_C-1$, while the auxiliary indices begin at $N_T+N_X$ and end at $N_T+N_C+N_X-1$.
Their state equations are:
\begin{eqnarray}
   \label{eq:coll-node-dT}
   \rho_w C v_C \dot{T}_i &=& \Q{amb} + \Q{$C$, mflow} + \Q{$C$, external}
   \\
   \label{eq:aux-node-dT}
   \rho_w C v_X \dot{T}_i &=& \Q{amb} + \Q{$X$, mflow} + \Q{$X$, external}
\end{eqnarray}

\subsection{Heat flows}

The ambient terms are simple equations accounting for the surface area of an element and its current temperature relative to the exterior ambient temperature.
\begin{equation}
   \label{eq:Q-amb}
   \Q{amb} = U_s (T_a - T_i)
\end{equation}
Note that there are three $U_s$ values, one each for the tank surface, collector surface, and auxiliary heater surface.
The tank is assumed to have a uniform value of $U_s$ at each node, despite the larger surface area of the top and bottom nodes (which include the ends of the tank cylinder).
The calculation of the tank's $U_s$ value takes into account the total surface area of the tank and averages that head loss over all nodes.
This is based on the assumption that heat loss to the ambient acts on the same timescale as buoyancy of the water, so even if the top node cools more rapidly than the second-top, hotter water will rise to replace the loss.
This effect is not explicitly accounted for due to the faster action of heating and cooling due to mass flows through the tank.

The mflow energy flow accounts for the heat that enters and exits each node carried by the actual volume of water that flows through it.
As water is flows through the system during charging (hot water entering from the solar collector) drawing (hot water exiting to the load), and auxiliary heating, the composition of each node changes.
The factor $\dot{m}_i$, defined in \autoref{eq:mdot}, refers to the amount of water entering node $i$ from node $i+1$, the node above it, and is therefore positive when water is flowing downwards through the tank (in the direction of drawing).
In the collector and auxiliary loop, water flows from the entrance to the exit (as illustrated in \autoref{fig:discretised-tank}).
\begin{equation}
   \label{eq:tank-Q-mflow}
   \Q{$T$, mflow} = \max \left\{ 0, \dot{m}_i \right\}     C_i (T_{i+1} - T_i)
             + \min \left\{ 0, \dot{m}_{i-1} \right\} C_i (T_i - T_{i-1})
\end{equation}
We additionally define $\dot{m}_{N-1} = \dot{m}_{-1} = 0$ to handle the ends of the tank.
In the collector and auxiliary loop, the water flow is identical across all nodes, equal to $\dot{m}_c$ and $\dot{m}_x$ respectively.
These variables are `inputs' in the sense that they are controlled, but this control is usually performed by internal actuators instead of by external command.
For more detail see \autoref{sec:models:system:control}.
The load flow is a disturbance, and is named $\dot{m}_l$.

The heat flow from each inlet is calculated in terms of its current mass flow and temperature.
$\Q{inlet}$ represents the heat gained due to flows of water entering the tank from the collector, auxiliary heater, and load.
\begin{equation}
   \label{eq:Q-inlet}
   \Q{inlet} = \B{coll} \dot{m}_c (T_c - T_i)
             + \B{load} \dot{m}_l (T_l - T_i)
             + \B{aux} \dot{m}_x (T_x - T_i)
\end{equation}
The control functions $\B{*}$ are described in the next section.
The two inlet temperatures for collector and auxiliary heat, $T_c$ and $T_x$, are determined by taking the current temperature at the `outlet' nodes of the collector loop and auxiliary loop, $T_{i=N_T+N_C-1}$ and $T_{i=N_T+N_C+N_X-1}$.
The load inlet temperature $T_l$ is a disturbance.

Within the collector and auxiliary heater, the $\Q{*, external}$ flows are defined by external disturbances --- either insolation, or auxiliary power input as appropriate.
The mass flow heating equations are less trivial, ut are easily defined in terms of the mass flows through the collector and auxiliary loops:
\begin{eqnarray}
   \Q{$C$, mflow} &=& \dot{m}_c C (T_{i-1} - T_i)
   \\
   \Q{$X$, mflow} &=& \dot{m}_x C (T_{i-1} - T_i)
\end{eqnarray}
Note that at the first element of the collector loop, $i-1$ is taken to refer to element 0, the bottom of the tank.
At the first element of the auxiliary loop, $i-1$ is taken to refer to element $N/2$, the missle of the tank, where the auxiliary outlet is connected.

\subsection{Control functions}

\Autoref{eq:Q-inlet} introduced the two \emph{control functions} $\B{coll}$ and $\B{load}$ used by \authors{Cristofari02}, which are binary indicator functions that determine which nodes receive hot and cold water flows from the inlets.
(There is an additional control function of my own addition $\B{aux}$ which works identically, but did not exist in the original paper.)
These control functions simulate the hot and cold water from the collector and load passing through stratification devices as they enter the tank, so that they are distributed to the water layer which most appropriately matches their temperature.
If the hot water tank has a temperature distribution such that for all nodes $i$ in the tank, $T_{i-1} \le T_i \le T{i+1}$, these functions only ever return 1 for a single node in the tank.
This is not assumed in my implementation of these algorithms, but in practise it does hold since all heat changes in the tank are caused by water flows or uniform cooling due to ambient losses.
They are defined as follows:
\begin{eqnarray}
   \label{eq:B-coll}
   \B{coll} &=& \begin{dcases*}
      1 & if $T_c > T_i$ and $i = N-1$ \\
      1 & if $T_{i+1} \ge T_c > T_i$ and $i < N-1$ \\
      0 & otherwise
   \end{dcases*}
   \\
   \label{eq:B-load}
   \B{load} &=& \begin{dcases*}
      1 & if $T_l < T_i$ and $i = 0$ \\
      1 & if $T_{i-1} \le T_l < T_i$ and $i > 0$ \\
      0 & otherwise
   \end{dcases*}
   \\
   \label{eq:B-aux}
   \B{aux} &=& \begin{dcases*}
      0 & if $i < O$ \\
      1 & if $T_x > T_i$ and $i = N-1$ \\
      1 & if $T_{i+1} \ge T_x > T_i$ and $i < N-1$ \\
      0 & otherwise.
   \end{dcases*}
\end{eqnarray}

The mass flow through each node is defined by \authors{Cristofari02} using sums of these collector functions to express the presence of an inlet above/below the current node being examined as
\begin{equation}
   \label{eq:mdot}
   \dot{m}_i = \dot{m}_c \sum_{j=i+1}^{N-1} \B{col}
             - \dot{m}_l \sum_{j=1}^{i-1} \B{load}
             + \dot{m}_{\text{aux}, i}.
\end{equation}
$\dot{m}_{\text{aux}, i}$ represents the mass flow due to the auxiliary heating loop described in \autoref{eq:mdot-aux}.
It is slightly more complex than the mass flows caused by the collector and load loops, because their outlets are fixed at the bottom and top of the tank respectively.
It is given by
\begin{equation}
   \label{eq:mdot-aux}
   \dot{m}_{\text{aux}, i} = \dot{m}_l \left( o_i - \sum_{j=1}^{i-1} \B{aux} \right)
\end{equation}
where $o_i$ is a binary factor that selects nodes above the auxiliary outlet:
$$
   o_i = \begin{dcases*}
      1 & if $i \ge O$ \\
      0 & otherwise.
   \end{dcases*}
$$

\subsection{Model validity}

My implementation of the model was not verified against empirical data.
As this work focuses on comparing control strategies on the system, efforts were not made to ensure that physical parameters were realistic, only plausible.
The work of \textcite{Cristofari02} was assumed to be valid for the purposes of simulating the system in this thesis.

\section{Mixed-tank control model}
\label{sec:models:control}

Efficient control using convex optimisation requires a linear model, as described in \autoref{sec:background:convex}.
This is achieved by taking a very simplified view of the system used by \authors{Halvgaard12} and explained in \autoref{sec:review:mpc:halvgaard}.
The entire tank is described as a single temperature, taken to represent the average temperature of all layers.
\Autoref{eq:tdot-halvgaard} describes the rate of change of this single temperature in response to the current state, input and disturbance conditions.
\begin{equation}
   \label{eq:tdot-halvgaard}
   \dot{T} = -\frac{UA}{mC}T + \frac{P \nu_x}{m} u_e +
      \left[ \begin{array}{cccc}
         -\frac{50}{mC} & \frac{1}{m} & \frac{1}{mC} & \frac{UA}{mC}
      \end{array} \right] \left[ \begin{array}{c}
         m_l \\ T_l m_l \\ I \\ T_a
      \end{array} \right].
\end{equation}
Note that this equation involves a constant 50.
This accounts for the bilinear effect of hot water leaving the top of the tank to be consumed by the user.
The true equation would use $T$ instead of a constant 50.
However, this would not produce a linear model.
For the same reason, the disturbance input includes the bilinear term $T_l m_l$.

Instead of using values of mass flow which are used in the simulation model (as in, for example, \autoref{eq:mdot}), this model uses values such as $m_l$ which represent the total mass added to the system in a time step.
These values are calculated by integrating the rates of mass flow over a given interval when required.

\authors{Halvgaard12} get around this problem by specifying disturbances in terms of energy loss, rather than flow rate.
They assume that a fixed amount of energy will be drawn by the user, regardless of the temperature of the water the user is drawing.
This constant value is equivalent to that assumption --- we assume that the energy lost will be equivalent to 50 degrees multiplied by the flow rate at that instant.
If the tank is hotter, the flow rate will be lower (as the user mixes the hot water with more cold mains water), and if the temperature is lower, more water will be drawn.
Whether this matches actual user behaviour is unverified.

\subsection{Linear optimisation problem}

This section describes a generic convex optimisation problem for linear state-space models which may be time-varying.
A linear time-varying system takes the following form:
\begin{eqnarray}
   \label{eq:continuous-xdot}
   \dvec{x}(t) &=& A(t) \vec{x}(t) + B(t) \vec{u}, \\
   \label{eq:continuous-y}
   \vec{y}(t) &=& C(t) \vec{x}(t)
\end{eqnarray}
where $\vec{x}$ represents the state vector, $\vec{u}$ the input vector, $\vec{y}$ the output vector, and $t$ the current time.
From now, all $t$ arguments will be dropped for clarity; it is understood that all matrices may vary in time, and that $\dvec{x}$ and $\vec{y}$ are functions of time.

The system is discretised in the usual manner for linear state-space systems, assuming a zero-order hold on input signals between each time step, forming a new set of discrete-time equations
\begin{eqnarray}
   \label{eq:discretise-A}
   A_d &=& e^{A \delta t}, \\
   \label{eq:discretise-B}
   B_d &=& A^{-1} (A_d - I) B, \\
   \label{eq:discrete-xdot}
   \vec{x}_{t + \delta t} &=& A_d \vec{x} + B_d \vec{u},
\end{eqnarray}
though \autoref{eq:continuous-y} is unchanged.

In this simple model, we will include an explicit handling of `disturbance' inputs.
These refer to any system inputs which we do not have direct control over.
In the case of a water heating tank, disturbances include heat loss to the environment, heat contribution from solar collector, and the user load schedule.
Our only non-disturbance input is, of course, the electric heating element's state.

To make this distinction between disturbance and non-disturbance inputs, we decompose $B$ and $\vec{u}$ into
\begin{eqnarray}
   B &=& \left[\begin{array}{cc}
      B_u & B_w
   \end{array}\right], \\
   \vec{u} &=& \left[\begin{array}{c}
      \vec{u}_e \\ \vec{u}_w
   \end{array}\right]
\end{eqnarray}
where $\vec{u}_e$ represents our \emph{explicit} control signal to the system and $\vec{u}_w$ represents the disturbance input we cannot control.
When $B$ is discretised according to \autoref{eq:discretise-B}, we can simply reconstruct the discretised $B_{u, d}$ and $B_{w, d}$ from the appropriate columns of $B_d$.

Once the system has been described in this form, a planning problem can be formulated.
Following the general format of \autoref{eq:mpc-opt}, we must define the input and disturbance vectors over the planning horizon, and formulate the matrices that encode the system behaviour to constrain the output.
We use hatted vectors to denote vectors over the time horizon, i.e. whose values are predicted at some future interval.
For example, while $\vec{u}_w$ refers to the disturbance input vector at some instant, $\hvec{u}_w$ refers to the stacked vector of all predicted disturbances over the time horizon.

We consider a plan over a time horizon $H$ elements long (so the total look-ahead time will be $H \delta t$, where $\delta t$ is the planning resulution) where $H$ is of course a unitless integer.
First define the input vector over the time horizon
\begin{eqnarray*}
   \hvec{u} &=& \left[\begin{array}{c}
      \hvec{u}_{e, 0} \\ \hvec{u}_{w, 0} \\
      \vdots \\
      \hvec{u}_{e, H-1} \\ \hvec{u}_{w, H-1}
   \end{array}\right]
\end{eqnarray*}
which is the vertically-stacked control and disturbance signals for the duration of the entire time horizon.
Then the output $\hvec{y}$, which is similarly defined as the stacked vector of all output vectors over $H$, can be defined in terms of these vectors multiplied by some time-varying matrices that define the system behaviour:
\begin{eqnarray}
   \label{eq:mpc-vectors}
   \hvec{y} = \Psi \vec{x}_0 + \Theta \hvec{u}
\end{eqnarray}
given $\vec{x}_0$, the current state of the system when the plan is made.

The matrices $\Psi$ and $\Theta$ are defined as follows:
\begin{eqnarray}
   \label{eq:mpc-psi}
   \Psi &=& \left[\begin{array}{c}
      CA_d \\ CA_d^2 \\ \vdots \\ CA_d^H
   \end{array}\right]
   \\
   \label{eq:mpc-theta}
   \Theta &=& \left[\begin{array}{cccc}
      C B_d & 0 & 0 & 0 \\
      C A_d B_d & \ddots & 0 & 0 \\
      \vdots & \ddots & \ddots & 0 \\
      C A_d ^{H-1} B_d & \cdots & C A_d B_d & C B_d
   \end{array}\right]
\end{eqnarray}

\subsection{Problem instance}

Now we are ready to describe the specific state-space model, objectives, and constraints used in the mixed-tank controller.
The vectors $\vec{u}_e$ and $\vec{u}_d$, the explicit and disturbance inputs, are filled with the appropriate variables:
\begin{eqnarray}
   \vec{u}_e &=& \left[\begin{array}{c}
      u_e
   \end{array} \right]
   \\
   \vec{u}_w &=& \left[\begin{array}{cccc}
      m_l \\ T_l m_l \\ I \\ T_a
   \end{array} \right]
\end{eqnarray}
The system matrices $A$ and $B$ are also evident from \autoref{eq:tdot-halvgaard}:
\begin{eqnarray}
   A &=& \left[\begin{array}{c}
      -\frac{UA}{mC}
   \end{array} \right]
   \\
   B &=& \left[\begin{array}{ccccc}
      \frac{P \nu_x}{m} &
      -\frac{50}{mC} &
      \frac{1}{m} &
      \frac{1}{mC} &
      \frac{UA}{mC}
   \end{array} \right]
\end{eqnarray}

\subsection{Objective and constraints}

The remainder of the planning problem formulation concerns the objective function and constraints.
\Autoref{eq:mpc-opt} already specifies that the system dynamics, $\hvec{y} = \Psi \vec{x}_0 + \Theta \hvec{u}$, should be considered a constraint.

The objective function chosen, $f^*$, may have a significant effect on the behaviour of the controller.
However, the objective function is easy to write and modify thanks to the declarative nature of MPC.
The objective function decided upon for this thesis is the following:
\begin{equation}
   f^*(\hvec{y}, \hvec{u}) = |\vec{r}\transpose \min \left{ \hvec{y} - \vec{50}, 0 \right}| + \rho_w |\hvec{u}|_1
\end{equation}
The two terms of this equation respectively penalise deviation from the desired tank temperature and use of the control input.
The first term deserves some explanation.
The first expression of note is the use of the min function.
Given a reference average temperature of 50 degrees centigrade, I only wished to penalise negative deviations from this reference point.
I was happy to allow the water to be hotter than the specified setpoint, and trust the tank's internal controllers to ensure the water does not reach dangerous temperatures.
Therefore after subtracting the reference temperature, the elementwise min takes only those elements of the predicted output that are smaller than 0, i.e. those for which the temperature is less than the desired setpoint.

The value of 50 was chosen after inspecting the operation of the tank from simulations under thermostat control.
It was observed that the tank's average temperature typically set on the order of 10 degrees below its top temperature.
Therefore to produce a top temperature of 60 degrees, an average temperature or 50 degrees is demanded.

This new vector is then multiplied by the reference vector $\vec{r}$.
This vector is a binary constant of the appropriate dimension which contains 1 when the reference signal should be followed, and 0 when not.
This reference vector is created at the start of each planning problem to contain 1 elements only in hours when there is any espected user load.
This produces one of the key benefits of this predictive strategy.
Overnight, or during the middle of the day, the reference vector will contain zeroes, and hence the controller will not be penalised for letting the tank cool in those hours.
However, it \emph{will} be penalised for presenting a cooled tank to users in hours when there is expected to be demand, and may start to warm the tank in anticipation of those hours.

The only constraint on the controller is that $0 \le u \le 1$, as the control input is defined to mean the proportion of full power to be used by the auxiliary heater.
Obviously, negative power cannot be used, and nor can a greater amount than the heater's rated power.

There are no constraints placed on the absolute output value.
The controller has no way of \emph{removing} heat from the system, only the option of \emph{not adding} heat.
Given this limitation, specifying a maximum temperature constrains frequently leads to infeasible control problems in summer months, as the solar input provides enough heat to drive the tank to high temperatures with almost no actuation.
As noted above, the tank's internal controllers will prevent the auxiliary heater from achieving dangerous temperatures, so the controller need not take this constraint into account itself.

This results in the final convex optimisation problem solved at each control interval:
\begin{equation}
   \label{eq:controller-problem}
   \begin{aligned}
      & \underset{\hvec{u}}{\text{minimise}}
      & & |\vec{r}\transpose \min \left\{ \hvec{y} - \vec{50}, 0 \right\}| + \rho |\hvec{u}|_1 \\
      & \text{subject to}
      & & \hvec{y} = \Psi \vec{x}_0 + \Theta \hvec{u}, \\
      &&& \vec{0} \le \hvec{u} \le \vec{1}.
   \end{aligned}
\end{equation}

\section{Implementation}

The bulk of the work in this thesis involved formulating the above system models (particularly tank model) and implementing them correctly for simulation.
Python was chosen as the implementation language due to its large open-source ecosystem of libraries for numerical computing, particularly the SciPy~\cite{SCIPY} and NumPy~\cite{Walt11} packages.
In addition, the CVXPY library~\cite{CVXPY} provided a convenient interface for specifying DCP problems.

Before settling on Cristofari's tank model, a simulation was coded using the model described by \textcite{Pfeiffer11}.
However, the description in that paper was incomplete and I was unable to reproduce the model in a working state.
This incomplete source code is not listed.

The source code is organised around two major scripts which run the experiments for the thermostat and MPC controller.
These scripts are parameterised with command-line arguments to produce the various simulations used in model analysis.

The controllers developed were separated based on their functionality and implemented generically to support better code reuse.
For example, a generic thermostat controller was designed and reused several times for the internal tank controllers.
The generic implementation allowed controllers to be `stacked' or `nested' within each other to provide more complex behaviour.
For example, the MPC controller developed must perform PWM modulation of its output to convert from a continuous input signal to the binary control expected by the heating element.
This was achieved by inserting an MPC control function inside the PWM controller, which decided when to call the MPC routine, and how to deal with its output.

The Python source code that was developed throughout this thesis, and the scripts that produce the graphs displayed, is partially listed in \autoref{app:code}.
Those wishing to view the full source code or reproduce the experiments should visit \url{https://github.com/eightyeight/thesis}.
