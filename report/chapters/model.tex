\chapter{Models}
\label{ch:models}

This chapter introduces the various models used in both simulation of the system, and control.
\Autoref{sec:models:simulation} introduces the full nonlinear tank model which is used when running simulations.
\Autoref{sec:models:control} outlines the linear state-space model of the stratified tank for MPC, and then describes the simplifying assumptions made to express the nonlinear simulation model as a linear model.
These sections all use the variables defined in \autoref{tab:model-symbols}.

\begin{table}
   \caption{Symbols used in model equations}
   \label{tab:model-symbols}
   \begin{center}
   \begin{tabular}{l l}
      \toprule
      $N$ & Number of nodes in the discretised tank model \\
      $T_i$ & Temperature of node $i$ in the water tank \\
      $T_c$ & Temperature of hot water from the collector entering the tank \\
      $T_x$ & Temperature of hot water from the auxiliary heater entering the tank \\
      $T_l$ & Temperature of the mains water entering the tank \\
      $T_a$ & Temperature of the air outside the tank \\
      $T_T$ & Target temperature for hot water inlets \\
      $\Q{*}$ & Energy inflow into tank node $i$ caused by * \\
      $\B{*}$ & Control/indicator function for flow * at node $i$ \\
      $\rho$ Density of water
      $C$ & Thermal capacitance of water \\
      $v_T$ & Volume of a node in the tank \\
      $U_{s, i}$ & Surface conductivity of node $i$ \\
      \bottomrule
   \end{tabular}
   \end{center}
\end{table}

\section{Simulation model}
\label{sec:models:simulation}

A hot water tank is a complex nonlinear system when stratification is considered.
It is desirable to develop an accurate simulation of the system in order to test the validity of controllers designed with simplified models.

The popular multinode tank model, as discussed in \autoref{sec:review:stratified-tank-models}, treats the tank as a stack of identical vertical disks with their own heat flow equations.
I will reproduce this model, relying on the description in \textcite{Cristofari02} with my own augmentation to handle the auxiliary heater, which was not present in that work.
Note that this approach assumes stratification baffle devices are present on the solar and mains inputs into the tank, and that these devices work perfectly.

The Python source code implementing this tank model is found in \autoref{app:code:models:cristofari}.

\subsection{Discretisation}

The tank is split into $N_T$ identical horizontal disks as illustrated in \autoref{fig:models:discretised-tank}, numbered $0$ (bottom) to $N_T-1$ (top).
\todo{Badly needs diagram}
Each disk, called a node, has uniform volume $v_T$.
The heat flows into the node, as shown in \autoref{fig:models:discretised-disk}, are governed by the equation:
\begin{equation}
   \label{eq:tank-node-dT}
   \rho C v_T \dot{T}_i = \Q{amb} + \Q{$T$, mflow} + \Q{inlet}
\end{equation}
That is, the total energy flow through the node can be broken into three parts: ambient losses, changes from the mass flow through the tank while charging/drawing, and input from the three heat/cold inlet: the collector inlet, mains inlet, and auxiliary inlet.

The collector and auxiliary heater are also treated as discretised volumes of water with $N_C$ and $N_X$ nodes of volumes $v_C$ and $v_X$ respectively.
The indices $i$ of the collector nodes begin at $N_T$ and end at $N_T+N_C-1$, while the auxiliary indices begin at $N_T+N_X$ and end at $N_T+N_C+N_X-1$.
Their state equations look like this:
\begin{eqnarray}
   \label{eq:coll-node-dT}
   \rho C v_C \dot{T}_i &=& \Q{amb} + \Q{$C$, mflow} + \Q{$C$, external} \\
   \label{eq:aux-node-dT}
   \rho C v_X \dot{T}_i &=& \Q{amb} + \Q{$X$, mflow} + \Q{$X$, external} \\
\end{eqnarray}

\subsection{Heat flows}

The ambient terms are simple equations accounting for the surface area of an element and its current temperature relative to the exterior ambient temperature.
The parameter $U_{s, i}$ accounts for the different surface areas of the nodes at each end of the tank.
\begin{equation}
   \label{eq:Q-amb}
   \Q{amb} = U_{s, i} (T_a - T_i)
\end{equation}

The mflow energy flow accounts for the heat that enters and exits each node carried by the actual volume of water that flows through it.
As water is physically flowing through the system during charging (hot water entering from the solar collector) and drawing (hot water exiting to the load), the composition of each node changes.
The factor $\dot{m}_i$, defined in \autoref{eq:mdot}, refers to the amount of water entering node $i$ from node $i+1$, the node above it, and is therefore positive when water is flowing downwards through the tank (in the direction of drawing).
In the collector and auxiliary loop, water flows from the entrance to the exit (as illustrated in \autoref{fig:tank-nodes}).
\begin{equation}
   \label{eq:tank-Q-mflow}
   \Q{$T$, mflow} = \max \left\{ 0, \dot{m}_i \right\}     C_i (T_{i+1} - T_i)
             + \min \left\{ 0, \dot{m}_{i-1} \right\} C_i (T_i - T_{i-1})
\end{equation}
We additionally define $\dot{m}_{N-1} = \dot{m}_{-1} = 0$ to handle the ends of the tank.
In the collector and auxiliary loop, the water flow is identical across all nodes, equal to $\dot{m}_c$ and $\dot{m}_x$ respectively.
These variables are `inputs' in the sense that they are controlled, but this control is usually performed by internal actuators instead of by external command.
For more detail see \autoref{ch:model:control-internal}.
The load flow is a disturbance, and is named $\dot{m}_l$.

The heat flow from each inlet is calculated in terms of its current mass flow and temperature.
$\Q{inlet}$ represents the heat gained due to flows of water entering the tank from the collector, auxiliary heater, and load.
\begin{equation}
   \label{eq:Q-inlet}
   \Q{inlet} = \B{coll} \dot{m}_c (T_c - T_i)
             + \B{load} \dot{m}_l (T_l - T_i)
             + \B{aux} \dot{m}_x (T_x - T_i)
\end{equation}
The control functions $\B{*}$ are described in the next section.
The two inlet temperatures for collector and auxiliary heat, $T_c$ and $T_x$, are determined by taking the current temperature at the `outlet' nodes of the collector loop and auxiliary loop, $T_{i=N_T+N_C-1}$ and $T_{i=N_T+N_C+N_X-1}$.
The load inlet temperature $T_l$ is a disturbance.

\subsection{Control functions}

\Autoref{eq:Q-inlet} introduced the two \emph{control functions} $\B{coll}$ and $\B{load}$ used by \authors{Cristofari02}, which are binary indicator functions that determine which nodes receive hot and cold water flows from the inlets.
These control functions simulate the hot and cold water from the collector and load passing through stratification devices as they enter the tank, so that they are distributed to the water layer which most appropriately matches their temperature.
They only ever return 1 for a single node in the tank.
They are defined as follows:
\begin{eqnarray}
   \label{eq:B-coll}
   \B{coll} &=& \begin{dcases*}
      1 & if $T_c > T_i$ and $i = N-1$ \\
      1 & if $T_{i+1} \ge T_c > T_i$ and $i < N-1$ \\
      0 & otherwise
   \end{dcases*}
   \\
   \label{eq:B-load}
   \B{load} &=& \begin{dcases*}
      1 & if $T_l < T_i$ and $i = 0$ \\
      1 & if $T_{i-1} \le T_l < T_i$ and $i > 0$ \\
      0 & otherwise
   \end{dcases*}
   \\
   \label{eq:B-aux}
   \B{aux} &=& \begin{dcases*}
      0 & if $i < O$ \\
      1 & if $T_x > T_i$ and $i = N-1$ \\
      1 & if $T_{i+1} \ge T_x > T_i$ and $i < N-1$ \\
      0 & otherwise.
   \end{dcases*}
\end{eqnarray}
The mass flow through each node is defined by \authors{Cristofari02} using sums of these collector functions to express the presence of an inlet above/below the current node being examined as
\begin{equation}
   \label{eq:mdot}
   \dot{m}_i = \dot{m}_c \sum_{j=i+1}^{N-1} \B{col}
             - \dot{m}_l \sum_{j=1}^{i-1} \B{load}
             + \dot{m}_{\text{aux}, i}.
\end{equation}
$\dot{m}_{\text{aux}, i}$ represents the mass flow due to the auxiliary heating loop described in \autoref{eq:mdot-aux}.
Note again that the flow is positive in the direction of the collector loop (downwards).
\begin{equation}
\end{equation}
Finally, we must define the mass flow through a node due to the operation of the auxiliary loop.
It is slightly more complex than the mass flows caused by the collector and load loops, because their outlets are fixed at the bottom and top of the tank respectively.
It is given by
\begin{equation}
   \label{eq:mdot-aux}
   \dot{m}_{\text{aux}, i} = \dot{m}_l \left( o_i - \sum_{j=1}^{i-1} \B{aux} \right)
\end{equation}
where $o_i$ is a binary factor that selects nodes above the auxiliary outlet:
$$
   o_i = \begin{dcases*}
      1 & if $i \ge O$ \\
      0 & otherwise.
   \end{dcases*}
$$

\subsection{Collector input}

\todo{This section needs to be written.}
It works basically the same as the aux input.

\subsection{Model validity}

My implementation of the model was not verified against empirical data.
As this work focuses on comparing control strategies on the system, efforts were not made to ensure that physical parameters were realistic, only plausible.

\section{Mixed-tank control model}
\label{sec:models:control}

Efficient control using convex optimisation requires a linear model.
This is achieved by taking a very simplified view of the system used by \authors{Halvgaard12} and explained in \autoref{sec:review:mpc:halvgaard}.
The entire tank is described as a single temperature, taken to represent the average temperature of all layers.

\begin{equation}
   \label{eq:tdot-halvgaard}
   \dot{T} = -\frac{UA}{mC}T + \frac{P \nu_x}{m} u +
      \left[ \begin{array}{cccc}
         -\frac{50}{mC} & \frac{1}{m} & \frac{1}{mC} & \frac{UA}{mC}
      \end{array} \right] \left[ \begin{array}{c}
         m_l \\ T_l m_l \\ I \\ T_a
      \end{array} \right]
\end{equation}

\subsection{Linear state space model}

This is how we make a state-space model, consider the system as linear time-varying, but say it in a way that doesn't exclude the possibility that I might come up with other models, for example based on step response.
\todo{Fix these words.}
\begin{eqnarray}
   \label{eq:continuous-xdot}
   \dvec{x} &=& A(t) \vec{x} + B(t) \vec{u}, \\
   \label{eq:continuous-y}
   \vec{y} &=& C(t) \vec{x}
\end{eqnarray}
where $\vec{x}$ represents the state vector, $\vec{u}$ the input vector, $\vec{y}$ the output vector, and $t$ the current time.
From now, all $t$ arguments will be dropped for clarity; it is understood that all matrices may vary in time, and that $\dvec{x}$ and $\vec{y}$ are functions of time.
This system is discretised in the usual manner for linear state-space systems, assuming a zero-order hold on input signals between each time step, forming a new set of discrete-time equations
\begin{eqnarray}
   \label{eq:discretise-A}
   A_d &=& e^{A \delta t}, \\
   \label{eq:discretise-B}
   B_d &=& A^{-1} (A_d - I) B, \\
   \label{eq:discrete-xdot}
   \vec{x}_{t + \delta t} &=& A_d \vec{x} + B_d \vec{u},
\end{eqnarray}
though \autoref{eq:continuous-y} is unchanged.

In this simple model, we will include an explicit handling of `disturbance' inputs.
These refer to any system inputs which we do not have direct control over.
In the case of a water heating tank, disturbances include heat loss to the environment, heat contribution from solar collector, and the user load schedule.
Our only non-disturbance input is, of course, the electric heating element's state.

To make this distinction between disturbance and non-disturbance inputs, we decompose $B$ and $\vec{u}$ into
\begin{eqnarray}
   B &=& \left[\begin{array}{cc}
      B_u & B_d
   \end{array}\right], \\
   \vec{u} &=& \left[\begin{array}{c}
      \vec{u}_e \\ \vec{u}_d
   \end{array}\right]
\end{eqnarray}
where $\vec{u}_e$ represents our \emph{explicit} control signal to the system.
When $B$ is discretised according to \autoref{eq:discretise-B}, we can simply reconstruct the discretised $B_{u, d}$ and $B_{d, d}$ from the appropriate columns of $B_d$.

\autoref{eq:mpc-opt}
\todo{Diagram of $\hvec{y}, \vec{u}$ over planning horizon.}
This time horizon, $H$, is a unitless integer that determines how many time steps into the future the planning problem will consider.
We first define the output, input and disturbance vectors used in the planning problem, in terms of the time horizon:
\begin{eqnarray}
   \label{eq:mpc-vectors}
   \hvec{y} = \left[\begin{array}{c}
      \vec{y}_0 \\
      \vdots \\
      \vec{y}_H
   \end{array}\right],
   &
   \vec{u} = \left[\begin{array}{c}
      \vec{u}_0 \\
      \vdots \\
      \vec{u}_H
   \end{array}\right],
   & \text{and} \quad
   \hvec{d} = \left[\begin{array}{c}
      \vec{d}_0 \\
      \vdots \\
      \vec{d}_H
   \end{array}\right].
\end{eqnarray}
We can now define the matrices $\Psi$ and $\Theta$:
\begin{eqnarray}
   \label{eq:mpc-psi}
   \Psi &=& \left[\begin{array}{c}
      CA \\ CA^2 \\ \vdots \\ CA^H
   \end{array}\right]
   \\\label{eq:mpc-theta-u}
   \Theta_u &=& \left[\begin{array}{cccc}
      CB_u & 0 & 0 & 0 \\
      CAB_u & \ddots & 0 & 0 \\
      \vdots & \ddots & \ddots & 0 \\
      CA^{H-1}B_u & \cdots & CAB_u & CB_u
   \end{array}\right]
   \\\label{eq:mpc-theta-d}
   \Theta_d &=& \left[\begin{array}{cccc}
      CB_d & 0 & 0 & 0 \\
      CAB_d & \ddots & 0 & 0 \\
      \vdots & \ddots & \ddots & 0 \\
      CA^{H-1}B_d & \cdots & CAB_d & CB_d
   \end{array}\right]
   \\\label{eq:mpc-theta}
   \Theta &=& \left[\begin{array}{cc}
      \Theta_u & \Theta_d
   \end{array}\right]
\end{eqnarray}

\subsection{Matrix construction}

The vectors $\vec{u}_e$ and $\vec{u}_d$ are defined as follows:
\begin{eqnarray}
   \vec{u}_e &=& \left[\begin{array}{c} \dot{m}_{\text{aux}} \end{array} \right] \\
   \vec{u}_d &=& \left[\begin{array}{cccc} \dot{m}_c & \dot{m}_l & T_a & T_l \end{array} \right]
\end{eqnarray}

