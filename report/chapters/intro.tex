\chapter{Introduction}

\section{Aim}

The objective of this thesis is to evaluate designs for a controller for the heating element of a domestic hot water tank connected to a solar collector which is able to reduce the weekly energy use required to satisfy the users' comfort requirements, when compared to a thermostat controller or custom boosting schedule.

\todo{Mention the use of linear and nonlinead MPC, and maybe DCP.}

\section{Motivation}

\section{Scope}

\begin{itemize}
	\item Evaluate current common controller designs
	\item Evaluate literature on control of SDHW systems
	\item Implement simulation of hot water tank with solar collector and booster (validation?)
	\item Implement simple controller
	\item Implement smart controller
	\begin{itemize}
		\item Mixed-tank simplification
		\item Stratified simplification?
		\item Linearisation
		\item Nonlinear MPC?
	\end{itemize}
	\item Evaluate differences
	\begin{itemize}
		\item Energy use
		\item Bill
		\item Comfort
	\end{itemize}
	\item Evaluate robustness
	\begin{itemize}
		\item Prediction accuracy
		\item System scale
	\end{itemize}
\end{itemize}

\section{Approach and structure}

In \autoref{ch:background}, I will discuss the broad situation of domestic solar water heating with a particular focus on the Australian industry, the current state-of-the-art in their control, and the mathematical background of optimisation methods I will use in my controller.
In \autoref{ch:review}, I will outline relevant research, both historical and current as necessary.
I will also include potential applications and future directions of the theory and implementation of this thesis.
\todo{Awful phrasing. Learn 2 english}

\Autoref{ch:models} introduces the mathematical models that I use for water tank simulation and for the predictive controller.
It outlines their derivation, suitability for this task, and gives some details about their implementation in source code.

In \autoref{ch:evaluation} I will present the results of the experiments descirbed in \autoref{ch:experiments}, and evaluate the performance of the controllers under different objectives, disturbance conditions and constraints.
