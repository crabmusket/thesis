\documentclass{article}

\title{Honours Thesis Proposal}
\date{\today}
\author{Daniel Buckmaster \\ 310227488}

\begin{document}

\maketitle

\begin{tabular}{ll}
	{\bf Title:} & Model Predictive Control for Solar Heated Water Systems \\
	{\bf Proposer:} & Daniel Buckmaster \\
	{\bf Supervisor:} & Dr. Ian Manchester \\
\end{tabular}

\section{Background}

Current domestic and commercial solar thermal systems are often simply controlled
by differential or proportional control schemes* that cannot consider the complexities
of the underlying system. User load patterns, tank stratification, solar
irradiance prediction, and pump hunting are important factors to consider
when controlling a water heating system*. Ideally, the system would use as much
solar energy as possible to provide heated water, with secondary gas/electric
boosters only providing what is absolutely necessary to meet the user's load
schedule requirements.

Model-predictive control is used in a variety of industrial applications \cite{Camacho04}
and is a proven method to incorporate predictions and model information explicitly into
a control scheme. It is a natural fit for problems where load schedules may be
known ahead of time, and the process dynamics are known and well-approximated by
mathematical models.

\section{Problem statement}

Create a model-predictive control scheme that maximises solar hot water usage
in a hybrid domestic or light commercial installation.

\section{Method}

Mathematical models of the hot water storage tank, solar collector, and piping
arrangement will be derived. Convex relaxations of these equations will be used
to solve the optimal control problem, and the controller tested in simulation.
Historical data of solar water system use will provide samples of user load
patterns, and will also be used to verify/compare the final controller.

\section{Schedule}

\bibliographystyle{plain}
\bibliography{../report/library}

\end{document}
