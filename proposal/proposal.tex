\documentclass{article}

\title{Honours Thesis Proposal}
\date{\today}
\author{Daniel Buckmaster \\ 310227488}

\begin{document}

\maketitle

\begin{tabular}{ll}
	{\bf Title:} & Model Predictive Control for Solar Heated Water Systems \\
	{\bf Proposer:} & Daniel Buckmaster \\
	{\bf Supervisor:} & Dr. Ian Manchester \\
\end{tabular}

\section{Background}

\subsection{Solar water heating}

Current domestic and commercial solar thermal systems are often simply controlled
by differential or proportional control schemes* that cannot consider the
complexities of the underlying system. User load patterns, tank stratification,
solar irradiance prediction, and pump hunting are important factors to consider
when controlling a water heating system*. In addition, the potential growth of
legionella bacteria in warm water has resulted in legislation of specific heating
patterns that allow tank water to remain safe. Ideally, the system would use as
much solar energy as possible to provide heated water, with secondary gas/electric
boosters only providing what is absolutely necessary to meet the user's load
schedule requirements. The question of {\it when} to boost is not trivially
determined.

\subsection{Model-predictive control}

Model-predictive control (MPC) is used in a variety of industrial applications
\cite{Camacho04} and is a proven method to incorporate predictions and model
information explicitly into a control scheme. It is a natural fit for problems
where load schedules may be known ahead of time, and the process dynamics are
known and well-approximated by mathematical models.

\subsection{Convex optimisation}

MPC requires solving an optimisation problem each sampling period to determine
the optimal course of action. Convex optimisation problems are a subclass of
optimisation problem for which a globally optimal solution can be determined
very efficiently on modern hardware, even for large-dimension problems. If a
problem is convex, or can be transformed easily into one that is, or can be
solved adequately by using the solution of a convex relaxation of the problem,
then it is feasible to apply MPC for optimal control.

An open problem in optimisation is efficiently solving problems where the decision
variables must take on binary values.

\section{Problem statement}

Create a model-predictive control scheme that maximises solar hot water usage
in a hybrid domestic or light commercial installation and adheres to regulation
for the safe operation of such systems.

\section{Method}

Mathematical models of the hot water storage tank, solar collector, and piping
arrangement will be derived. Convex relaxations of these equations will be used
to solve the optimal control problem, and the controller tested in simulation.
Historical data of solar water system use will provide samples of user load
patterns, and will also be used to verify/compare the final controller.

\section{Validation}

The system will be compared to existing technologies using STCs, or Small-scale
Technology Certificates, which represent the amount of electricity displaced by
the use of a solar water heating system. \cite{CER13} They provide a benchmark
against which different renewable technologies can be compared, as well as
providing monetary incentive for home or business owners to install these
technologies.

{\it Alternatively}



\section{Schedule}

\bibliographystyle{plain}
\bibliography{../report/library}

\end{document}
