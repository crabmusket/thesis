\documentclass{article}

\title{Honours Thesis Proposal}
\date{\today}
\author{Daniel Buckmaster \\ 310227488}

\begin{document}

\maketitle

\begin{tabular}{ll}
	{\bf Title:} & Model Predictive Control for Solar Heated Water Systems \\
	{\bf Proposer:} & Daniel Buckmaster \\
	{\bf Supervisor:} & Dr. Ian Manchester \\
\end{tabular}

\vspace{5mm}

{\it Note: instances of [] signify statements that require citation.}

\section{Background}

\subsection{Solar water heating}

Current domestic and commercial solar thermal systems are often simply controlled
by differential or proportional control schemes (\cite{LSTS}, for example \cite{Hasan11} and \cite{Cao14})
that cannot consider the complexities of the underlying system.
Important factors to consider when controlling a water heating system include
user load patterns [], tank stratification \cite{Hollands89}, solar irradiance
prediction [], and pump hunting \cite{LSTS}.
In addition, the potential growth of legionella bacteria in warm water has
resulted in legislation of specific heating patterns that allow tank water to
remain safe (for example, \cite{AS3666}).
Ideally, the system would use as much solar energy as possible to provide heated
water, with secondary gas/electric boosters only providing what is absolutely
necessary to meet the user's load schedule requirements.
The question of {\it when} to boost is not trivially answered.

\subsection{Model-predictive control}

Model-predictive control (MPC) is used in a variety of industrial applications
\cite{Allgower04} and is a proven method to incorporate predictions and model
information explicitly into a control scheme \cite{Camacho04}.
It is a natural fit for problems where load schedules may be known ahead of time,
and the process dynamics are known and well-approximated by mathematical models.

MPC has previously been applied to full-system climate control, yielding between
15\% and 30\% savings on energy bills in \cite{Siroky11} and up to 20\% in \cite{Ma12}.

\subsection{Convex optimisation}

MPC requires solving an optimisation problem each sampling period to determine
the optimal course of action.
Convex optimisation problems are a subclass of optimisation problem for which a
globally optimal solution can be determined very efficiently on modern hardware,
even for large-dimension problems.
If a problem is convex, or can be transformed easily into one that is, or can be
solved adequately by using the solution of a convex relaxation of the problem,
then it is feasible to apply MPC for optimal control.
Many well-known optimisation problems (such as least-squares) are convex.

An open problem in optimisation is efficiently solving problems where the decision
variables must take on binary values [].
This is of some interest in water heating systems with large boilers which have
only binary control options, especially since these systems cannot be rapidly
switched on and off [].

\section{Proposal}

\subsection{Problem statement}

Create a model-predictive control scheme that maximises solar hot water usage
in a hybrid domestic or light commercial installation and adheres to regulation
for the safe operation of such systems.

\subsection{Method}

Mathematical models of the hot water storage tank, solar collector, and piping
arrangement will be derived. Convex relaxations of these equations will be used
to solve the optimal control problem, and the controller tested in simulation.
Once the basic controller is working, it will be extended in an ad-hoc fashion
to perform boosting for legionella control.
Historical data of solar water system use will provide samples of user load
patterns, and will also be used to verify the final controller and compare it
with current technology.

\subsection{Validation}

The system will be compared to existing technologies using STCs, or Small-scale
Technology Certificates, which represent the amount of electricity displaced by
the use of a solar water heating system \cite{CER13}. They provide a benchmark
against which different renewable technologies can be compared, as well as
providing monetary incentive for home or business owners to install these
technologies.

\section{Schedule}

Literature review has begun and will be finished before Week 13 of Semester 1, 2014.
Mathematical model development will begin in Week 7 of Semester 1, and be complete
by Week 1 of Semester 2.
Controller design will begin in Week 11 of Semester 1 and be completed by Week 4
of Semester 2.
Simulation with CER data will take place before Week 5 of Semester 2, and will
be written up before Week 10 of that semester.

\clearpage

\bibliographystyle{plain}
\bibliography{../report/library}

\end{document}
